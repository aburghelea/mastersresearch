Although not really widely used in an official manner, people management techniques can be seen in other fields.
\subsection{Military Field}
\label{subsec:military}
The army has to always keep the morale of their troops high (especially in war theatres). In order to obtain this, they will sometimes set up shows and arrange leaves for the soldiers to visit their families. When the commanders prove that they care more about the troops instead of their own person there has been seen to improve the morale. One situation of this is described by Sinek in \cite{safe} about Cpt. William Swenson. Cpt Swenson preferred to rush in and retrieve his injured soldiers, putting himself in the middle of cross fire. In this way he proved that his troops were more important than his own safety. As a result, the soldiers that were returning fire were even more motivated to eliminate the threat fast.

\subsection{Legal Field}
\label{subsec:legal}
In law firms coaching is very visible, while a new employee usually start as an junior associate he is responsible with digging through the paperwork and find the proof that helps his case (or even loopholes in the laws). After a couple of years, after he has gather experience in the courtroom alongside senior colleagues he is promoted to an associate level. In this situation trust is given to him, because he is allowed to be a led attorney for some cases. His senior colleagues will monitor his performance and provide feedback. When a senior position is open, the most promising associate is promoted and know he becomes in charge of training junior associates.
