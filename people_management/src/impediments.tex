It may seem that management can be a simple recipe of actions, but things can sometimes turn in an unexpected way. A manager has to anticipate and learn how to tackle difficult situations.

\subsection{Setting Incorrect Development Objectives}
\label{subsec:devobjectives}
It is a well established fact int the IT industry that objectives that are not \textit{SMART} \cite{doran}  will easily be unachieved. The reason behind this is that if that they are neither specific or time boxed. Let's take the following example:
\textit{The objective is to increase the revenue}
In this case the objective is not \textbf{S}pecific because increasing the revenue can be considered if it was increased by 1\$ or by 10\% compared to the last year. From the same reason it is not \textbf{M}easurable and \textbf{A}chievable. Also it is not \textbf{R}elevant to all the employees (a developer can not directly increase the revenue). In the same time the objective is not \textbf{T}ime-bound because increasing the revenue can be done either in one day or ten years. A better example would be.
\textit{We will increase our revenue with 10\% compared to the last financial trimesters until the end of this year, by optimizing the application John Doe is developing in order to consume less cloud resources, thus reducing our infrastructure costs with 20\%.}

This example is specific: optimize the application, measurable: increase revenue by 10\%, achievable: by optimizing the application, relevant: for reducing the infrastructure closed, time-bound: until the end-of semester.

It is impossible for an employee to follow an objective that he does not understand and he does not accept. This is why most objectives are completed. In order for an employee to follow an objective it is important that the objective is personal. A person objective has to be tailored by the manager and agreed with the employee, and in order to be person it has to be specific to what he is working on and it has to add value to both him and the company.

\subsection{Interpersonal Conflicts}
\label{subsec:conflicts}
If an employee feels that his manager does not represent his interest he will refuse to work with him and even follow his objectives. This will lead to low performance and missing deadlines. Unfortunately if this case occurs the only viable solution would be to assign a new people manager to the new person in cause. Because this would require a new accommodation period for trust to settle between new manager and the employee, it is advisable to try to settle the difference by setting meeting between the employee, manager and supervisors. In some cases the issue may be caused by trivial cause that can be fixed, but in my personal experience I found out that the stubbornness of both manager and employee can get in the middle and a consensus can not be reached without resentment.
\subsection{Incorrect Delivery and Interpretation of Feedback}
\label{subsec:incorrectfeedback}
Feedback is usually a serious topic because the employee will receive both positive and negative feedback. It is important to deliver negative feedback and not attack the values of the receiver (e.g: You are always messing things up, causing the company to lose money) and use a \cref{subsec:fac} model. In the same time it is important that before the meeting is over to ensure that the employee received the correct message. If the wrong message is received, the employee will feel that he is only reproved. The manager must insure that the employee understood what he has to improve and also has some action items in order to help him in the improving process.

Impediments can not be avoided, but they can be overcome attention and correct attitude.
