People management can be tackled in different ways, depending on who is responsible for it. The main classification for people management is:
\begin{enumerate}
\item Human Resources responsibility
\item Team Leader responsibility 
\item Dedicated branch of people managers
\end{enumerate}

Each of this approaches have it's advantages but also disadvantages, that are detailed bellow:
\subsection{Dedicated Subdivision of the Human Resources Department}
\label{subsec:hrdep}

The human resources department is responsible for hiring, firing and legal paper-work related to employees but not their direct working field. In the same time, they could be in charge with ensuring enough motivation for the existing employees. Having people evaluated by the same department will help the company have coherency across the entire company by using the same guidelines. On the other hand evaluations about performance of each employee by persons who do not have direct experience in the field, may prove unfavourable for the employee. This comes from the fact that a technical employee may have delivered a product that is high quality, but with very few resources or resources that are very difficult to use, while in the same time another one could have delivered a product with the same quality but had access to more and better resources. In this case the first employee should be ranked higher than the second, but it can be very difficult for a HR representative to grasp the reason to why this should be. 

By the fact that an employee is evaluated by someone who has no direct experience in the working field (e.g.: programmer evaluated by a behavioural expert) we end up in the same situation, anonymously quoted only (and sometimes attributed to Einstein \cite{aeq}): 
\begin{displayquote}
``Everyone is a genius. But if you judge a fish by its ability to climb a tree, it will live its whole life believing that it is stupid''
\end{displayquote}

\subsection{Technical Person from Inside the Project Team}
\label{subsec:techin}

A way to accurately evaluate an employee, is to delegate the people management task to the technical team leaders. The team leader, is usually an active person inside the project, with high visibility over the tasks and progress. The evaluation will be done very easily with accurate contextual information about the employees actions. On the other hand the team leader has visibility over one project, and depending on his standards he may provide harsh evaluations compared to other team leaders, thus not having coherency on the company levels. In the same time, the team lead may need a member to work on tasks that the employee does not like and may demotivate him. 

Although the evaluations from the team leader can be more accurate, the entire management process may fall behind the project priorities. 

Personally I had to also manage an employee that was part of my team and the decisions for the project often clashed with the employees needs, making it difficult to find win-win solutions and ending up in taking even more workload and not delegating.

\subsection{Technical Person Outside the Project Team}
\label{subsec:techout}

To solve the problems from above, another possibility would be that the people manager be a technical person with experience in the same fields as his direct reports, but not directly involved in the same projects as the direct reports. This will help in making decisions without having conflicting interest like in the case of \cref{subsec:techin} and also avoiding the situation when the evaluator does not have the similar working experience with the evaluated person like the case from \cref{subsec:hrdep}. The drawback would be the fact that a decision from the people manager could be in direct contradiction with the project needs.

For example: Employee John Doe does not want to work with drools technologies, but he is the only person who can tackle the difficult task needed for the next release. From the team-leaders point of view, he needs to continue working with drools, but from a motivational point of view, he should work on other sides of the project. This decisions are mutually exclusive. In this situation the lead, manager and employee have to find a consensus about both motivating the employee and delivery. One possible solution would be to agree that John will finish the tasks necessary for the next release, and then he can start working on different modules. In the mean time, the team leader has to set up a knowledge transfer program in order to train other members in the zone that only John has experience. The role of the manager here is visible in convincing the employee that he will work with drools for just a short period of time and to motivate him to finish as soon as possible. In my case, the best motivator was to promise (and also keep the promise) that if he finishes the tasks early and trains his colleagues, he can start working on the new modules earlier than the release is planned. In this way the employee would receive a confidence boost and also will be determined to finish faster. In the same time he will ensure that his results will have a high degree of quality, because he will not wish to return to the old tasks if they do not perform as expected.

By involving the manager, lead and employee in finding the solutions, the decision making was not unilateral and it was also proved in the military field by D. Marquet in leading the Santa Fe and is described in \cite{turn-that-ship-around} by:
\begin{displayquote}
``What can we do so that you can actually run this ship?''
\end{displayquote}
