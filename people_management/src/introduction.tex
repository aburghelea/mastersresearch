Employees are the biggest asset a company can have and in the IT industry they can make the difference between the rise and fall of a business. 
The employees attitude and performance can easily make the difference between a \SI{1}[\$]{M} company and a \SI{10}[\$]{M} company.

One of the trickiest parts of a manager is people management, which means that he is in charge of motivating, training, courages and even defends the employees if the situation requires it. In the same time the manager is also in charge of firing, disciplining and evaluating his employees. Although this responsibilities usually seem to be at in opposition, the integration of both positive and negative aspects makes the difference between a successful and unsuccessful manager that is able to create a good and productive environment.

Although employees are considered human resources, they are not inventory and according to H. Ross Perot:
\begin{displayquote}
``People cannot be managed. Inventories can be managed, but people must be lead.''
\end{displayquote}
This already hints that the naming of \textit{people management} is a poor choice of words, but the responsibilities of a person in this position start to migrate from the management sphere into the leadership sphere.

\subsection{People Management Concept}
\label{sub-sec:pmconcept}

In big organizations, usually there is a hierarchy of management that helps keep the company running smoothly. Being a good manager is one of the toughest and sometimes the least acknowledged jobs. A manager has to both lead by example and also manage expectations. The management positions come in a lot of flavours varying from project management, branch management and ending with people management. 

Managers know that the critical difference between success and failure is made by the people. In the same time, surprisingly, there is little research demonstrating the link between people management and business performance. 

Although each company has it's own flavour of people management, it is very wide spread and accepted that a people manager is a person that manages one or more individuals and has the following attributions:
\begin{itemize}
\item hiring
\item creating job descriptions
\item creating and advising on career planes
\item performance evaluation
\item firing
\item receiving resignations
\end{itemize}

Overall the people manager job is a pragmatic and results oriented type of work where a person has to find the correct balance between company performance and employee satisfaction.

It usually has nothing to do with being a human resources person, but usually helps to have some HR knowledge (eg: legal notice period). 

Although a people manager usually focuses on the motivating and increasing the performance of the employees, they have to be aware that difficult situations (eg: firing) often arise and always keep in mind Donald Trumps quote:

\begin{displayquote}
``It's nothing personal. It's just business.''
\end{displayquote}

\subsection{Motivation}
\label{sub-sec:Motivation}

From June 2014 and until January 2015 I had the occasion to really get of my comfort zone because I was in a position of managing employees. The position I was in was very similar to \ref{subsec:techout}, having 4 direct reports outside of the technical team, but the difficult part came from the fact that that I also had a direct report from inside the project team, \ref{subsec:techin}. The trickiest part came from the fact that was also the team leader and the my direct report was also severely demotivated and did not want to do the same things in the project that she had been doing for the last months. On one side I had to make decisions that would favour the project but on the other hand I also had to factor into account her needs. It may seem like a simple solution to switch the owners of two modules, this was simply not possible because the other members were had more seniority and their skills were needed in other areas of the project, areas where the person in cause could not handle within the hard deadline of the team.

The motivation behind this report is to give an incentive over what a people manager role means and how to overcome some difficult situations that may arrive. My former manager once said about people manager that:
\begin{displayquote}
``They are the persons who wear two hats: a hat the represents the needs of the company and one that represents the desires of the employees.
\end{displayquote}
The most widespread difficulty that I observed among my people manager colleagues, especially the newly appointed ones was that most of us were not able to wear both the hats, dividing us into two camps: the people representatives (acting as a syndicate) and the company rulers (acting as a board of decision enforces).

\subsection{Objectives}
\label{sub-sec:objectives}


The objective of this paper is to propose solutions to tricky situations encountered as people manager. The second objective of this study is to elaborate a questionnaire based on my personal decision in coaching my direct reports in order to validate the correctness of my actions.