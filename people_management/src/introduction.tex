Employees are the biggest asset a company can have and in the IT industry they can make the difference between the rise and fall of a business. 
The employees attitude and performance can easily make the difference between a \SI{1}[\$]{M} company and a \SI{10}[\$]{M} company.

One of the trickiest parts of a manager is people management, which means that he is in charge of motivating, training, courages and even defends the employees if the situation requires it. In the same time the manager is also in charge of firing, disciplining and evaluating his employees. Although this responsibilities usually seem to be at in opposition, the integration of both positive and negative aspects makes the difference between a successful and unsuccessful manager that is able to create a good and productive environment.

Although employees are considered human resources, they are not inventory and according to H. Ross Perot:
\begin{displayquote}
`` People cannot be managed. Inventories can be managed, but people must be lead.
\end{displayquote}
This already hints that the naming of \textit{people management} is a poor choice of words, but the responsibilities of a person in this position start to migrate from the management sphere into the leadership sphere.
\subsection{People Management Concept}
\label{sub-sec:pmconcept}

\todo{write intro} \newline

\subsection{Motivation}
\label{sub-sec:Motivation}

\todo{Write motivation} \newline

\subsection{Objectives}
\label{sub-sec:objectives}

\todo{write objectives}

