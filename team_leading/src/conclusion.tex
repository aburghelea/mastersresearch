% vim: set tw=78 sts=2 sw=2 ts=8 aw et ai:
Based on the details above, a leader has to evaluate his decisions and ask himself what can he do to correct the situation. While a leader coordinates a team he has to take into account what is the time frame that he has until project delivery is needed. This is important in order to decide if a \textit{Leader-Follower} or \textit{Leader-Leader} approach is more suited. It's the leaders responsibility to create slack time so that he creates a good environment for the team to learn new skills and he also has to take into account that his success is a direct result of the teams efforts. While leading a team it is important to delegate part of the tasks so that each team member has the occasion to get a clear picture of what leadership means. In the same time it is important to create personal relationships with the members so that they can be motivated accordingly and also to avoid a high turn-over rate.

\subsection{Questions a leaders has to answer in order to validate his leadership approach}
In team coordination there is no fool proof recipe for success, in the same time a leader can evaluate his actions on a high level. Every team leader should be able to know how to answer the following questions:

\begin{enumerate}
\item \textbf{What kind of leader do I want to be? (Technical decision maker or Team leader/Coach)} The leader has to know what are his expectations from himself.
\item \textbf{What is the lifespan of the team?} If a leader will coordinate a team for a single and short-lived project it may not be feasible to create slack time.
\item \textbf{Do I know each member in person and how he thinks? If not how do I obtain this information?} Without really knowing each team member a leader will not know how to correctly motivate each individual.
\item \textbf{What phase is the team in?} Creating a self organizing team can not be achieved by burning phases. A team has to go through day-by-day phase and then a learning phase in order to truly become self-organizing.
\item \textbf{If I suddenly leave the team, will it do better or worse?} If the answer is worse and the goal is to have a self organizing team then it means that the leadership is either incorrect or incomplete
\item \textbf{How can I teach my team to not fear failure} Leadership is often done by example. If the leader proves that he is able to risk even if the repercussions could be devastating.
\item \textbf{What tasks can I delegate and to whom} Not any task on the leaders plate can be delegated. A leader must correctly identify the person that is both willing and wishing to do his tasks?
\item \textbf{Does my team have enough time to learn?} Without developing the skills of a team progress is absent. Without progress de-motivation sets in.
\item \textbf{What would you and your team like to accomplish?} The project is not the answer to this question. To ensure the delivery on time, the team has to have a common goal.
\item \textbf{Are you unintentionally solving the teams problems instead of teaching them how to do it?} Coaching is the most important part of leadership. If a leader solves the problems of the team without teaching them how to solve them themselves he is not a leader, he is a commander.
\end{enumerate}

It is important to observe that there is no question about how the leader is seen by the team. It is important for a leader to be trusted, but it is not important to be liked.

\subsection{Future work}

For future development this paper will be enhanced by a study about team coordination and successful proven techniques in the Romanian industry. The study will be conducted by direct interviews and questionnaires with both team leaders and team members to observe the rate of success of common decisions.Also a comparison between the teams expectations and management expectations will be.

In order to develop a more in-depth vision about team leadership, the position will be studied form a point of view where a team leader has to set expectations from his team while coordinating with the senior management in order to ensure company success. In the same time it is important to understand what are the expectations that a team members have from a team leader and what makes a difference between a good leader and a bad leader. It will also be studied what if a good leader is equal to a successful leader.

Team leadership is in general a hard task to do and usually the results are not always obvious. In order to create a self organizing team a leader has to continuously improve his management skills along side his technical skills. Decisions should be made based on both previous experience and also on team input. There is no guaranteed method for success and that is why there is always room for improvement. 