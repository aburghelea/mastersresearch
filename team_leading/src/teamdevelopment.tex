% vim: set tw=78 sts=2 sw=2 ts=8 aw et ai:

If a technical leader has to deliver a project in a short period of time (less than 4 months) chances are that he will not have enough time to develop the skills of the members or even let them grow, so he has to ensure that the team understands the constrain and he has to motivate the team by trusting them with task in difficult enough to challenging them but easy enough so that they can finish them on time. If the projects lasts more than 4 months, the team leader will have to help the team grow by ensuring that each member learns something new.

\subsection{Motivating teams based seniority level of the members}
Junior team members are the easiest members to motivate. The biggest motivator for a junior member is the fact that he is part of a team. If a junior member feels that he is seen by the rest of the team as a functioning unit, he will be motivated to learn and work for the project. The second factor for junior members is remuneration. From a junior members' point of view this is a win-win situation. He learns from more senior colleagues and he gets paid to do it. The junior members are always aware that the quantity and quality of their work they produce is lower than the work of their more experienced colleagues. A common de-motivator for junior members is routine. If they have to work on the same kind of tasks for a long periods of time, they will get bored and will think of leaving the company.

After we have finished the project with the first team (the interns) we asked them to tell us what they liked most. 90\% of the answers were: \textbf{diversity}. The team appreciated the fact that they had the opportunity to work on different areas of the application (integration, database, services, front-end). 

Mid-level developers start to have a good understanding of the language that they are using and usually prefer to brainstorm and pitch ideas to senior colleagues. They are aware that if they want to become seniors they have to learn from the seniors. While tackling a difficult design problem, they will often come up with ideas. This situation is usually hard to tackle because it can either inhibit the person in cause or it can boost his confidence level to take more risk. If his idea is dismissed with an argument like ``You don't have enough experience, let the other guys handle it'', the member in cause will be reluctant to take on new challenges or event express his opinion. If his idea is rejected with valid arguments that are explained to him or if his idea is accepted, his confidence level will rise and he will try to prove with every occasion that he is evolving constantly. They also like to be challenged to work on more complex use cases from an application business point of view.

Senior members are the hardest to motivate because they are aware that the positions that they can grow to are fewer that the number of seniors. They either want to become team leaders (to be read: fill you position) or to become software architects/coding gurus. Senior members like to be challenged, but in comparison to mid-level developers they like to be challenged from an software point of view. They like to have a general overview of the design and architecture of the application and do not care too much about the business logic. They are focused on performance issues, redundancy, integration issues and most importantly code quality. The senior members feel very appreciated when the team leader delegates tasks from his plate and onto theirs. This does not mean that the leader has to delegate all of his tasks (or better said what the senior members feel that are all the leaders tasks). Also it is very important for the leader to correctly chose what kind of tasks he delegates. Some members prefer to receive only technical decision tasks while others prefer to receive managing tasks (e.g.: filling reports, reviewing solutions, presenting the project to the customer).

Personally, in my desire to show a senior member that team leadership is a great learning opportunity for everyone combined with the fact that I did not take the time get to now him, I let him handle both design tasks alongside the team and also a big part of the internal team coordination (specifically task delegation). By the time I realised that he did not like to help coordinate the internal team and even design modules that he would not work on directly, he had already made the decision to leave the company. By trying to motivate a member without knowing what he likes, I have managed to lose a dedicate programmer with great technical skills.

Motivation is a tool to be used with custom approaches tailored to each member. Motivation does not have a simple recipe for success, it has a very complicated one and the key to solving stands in knowing (truly knowing, not just his CV) the person in front of you.

\subsection{Helping team members develop their skills}
A team leader has to acknowledge the fact that he is not the person that will deliver the project, he is just the one that will ensure that the project is delivered. The project will be delivered by the team and he has to facilitate this. From personal experienced, I have learned that any long project does not start with the necessary experts for delivering the project without hinges. None of the projects that I have worked on were started by teams that have all the necessary knowledge to tackle any task. For each project we had to learn new skills, new apis and new approaches. But before we learned the hard-skills, we were taught how to risk.

In order to learn new skills, any person has to leave it's comfort zone. Learning in the professional environment is totally different than the academic. Every individual has to decide what to learn and most important how to learn. He also has to experiment what he is reading. This fact usually causes uncertainty and fear of failure. A team leader often hears questions like:

\begin{itemize}
\item What if this framework will not help us in solving our problem?
\item What if we take to learn and the management will question our performance?
\item How do we know if we are learning the correct thing?
\end{itemize}

The best answers I have heard are:
\begin{itemize}
\item We learn another one.
\item They don't fear that they are investing in us and we leave, they fear that if they don't invest we stay.
\item We try it in the project.
\end{itemize}

The best qualifier for the answers is not that ambiguous as it sounds. This kind of answers determined me to read and apply new things without having the guarantee that it will work. 

As soon as a team is ready to risk it can be challenged. The team leader has to tell the team what he needs for them know and let them decide how they will learn. But for the learning process to be truly effective, the team has to learn something difficult enough to get them out of the comfort zone. If they feel too comfortable enough when they are trying something new, it means that they already know how to obtain that result, but just with a different tool. 

Motivation and team development are the main way to ensure long term success. People don't leave companies, they leave their managers. If they do not feel motivated and supported to grow, they will easily leave the company.

The key concepts for motivating software developers are:
\begin{enumerate}
\item In depth knowledge of what each developers sees important for his career.
\item Developing the technical skills of the developers in a direction that fits both the desires of the company and the individual.
\item Demonstrating that failure results in a new start not a termination.
\item Making each opinion matter.
\end{enumerate}