% vim: set tw=78 sts=2 sw=2 ts=8 aw et ai:

Bruce Tuckman prosed in 1965 a team development model based on 4 stages and evaluated based on the relationship between the members. The stages of the Tuckman model are:
\begin{description}
\item[Forming] when the team is highly dependent on the leaders direction, they do not know their roles and there is no relationship between the members.
\item[Storming] when the members can't see eye to eye and conflicts are often because each one wants his idea to be used in the detriment of the other ideas
\item[Norming] when relationships between the members start to develop and consensus largely forms inside the team.
\item[Performing] when the team has a clear strategy and knows what to do and the team leader delegates and oversees the actions.
\end{description}

After going through these stages the team can be seen as it interacts and develops with three kind of interactions:
\begin{description}
\item [Day by Day] It survives day-by-day doing their tasks and they do not have any time free to learn new things
\item [Evolution] The team has time to learn new things in parallel with developing the new features
\item [Self-Organization] Each member of the team is capable to balance his tasks so that he is able to learn new skills and he chooses what to learn by himself
\end{description}

This phases are also depicted by Roy Osherove \cite{notes-to-a-software-team-leader} with the names: Survival, Learning, Self-Organizing

\subsection{Day by day stage}
For a team to be in a day-by-day stage an important requisite must not be present. This requisite is the time to learn. The team is focused only to deliver the tasks on their plate and the workload is big enough that they do not have any time to improve their personal skills. Although this could seem that the perfect stage from a management point of view and interpreted as "the team is working full-time, using the maximum amount of time to be productive" this can lead to an unmotivated team that in time will lose it's productivity and it will also lead to a high turn-over in the company. 

The leader of such a team has the duty to create slack time. It is mandatory for the management and client to understand the the current working speed will lost for a maximum of 2-3 months until the team members become frustrated and take into account leaving the company.

The leader must obtain the buy-in from the client and the management to create slack time. This slack time can be created by removing a part of the current commitments. In agile this is called a sprint scope change.

Although nobody will want to push the deadlines, experienced managers will understand that a demotivated team will not ensure delivery on time. Experienced managers agree that the key to success is a motivated team that is always helped to learn and develop new skills.

\subsection{Evolution stage}
A main goal for a team leader that focuses is to both get them in the Evolution phase and also get them out of this phase. If the team is in a day-by-day stage the lead has to get the buy-in of the stakeholders to create slack time and help the team learn new skills. If the team is in Evolution stage the lead has to gradually reduce the help he provides the members in the learning process in order to help them to migrate to the Self-Organizing phase. A good question every leader should ask himself is: ``Will my developers be better in a month or two than they were before? If not, how do I make that happen?"\cite{notes-to-a-software-team-leader}

A team can be considered to be in the evolution phase if both the following question is yes:
\begin{enumerate}
\item Does the team have enough free time to learn new skills?
\item Does the team use this free time?
\end{enumerate}
If the answer to both question above the free time is not really free, it's called slack time or professional day. 
During this phase, the leader has the opportunity to guide the members in learning new skills(e.g.: TDD, refactoring, new framework) and also 
guide them to be autonomous so that they can decide what things to learn.

\subsection{Self-Organizing stage}
This phase is the goal of all team leaders. To have a team that works so good that his expertise is not needed any more. If a team leader can leave work for a couple of days and not worry that the team can not go on without him, that's a hint that he coordinates a team in this phase. If he comes back and things have moved on without stalling or red flags being raised, he has a confirmation. In this phase the leader role is more of a facilitator and supervisor of the state of the team. It's recommended to get out of the teams way and let them do their job, because by now they know how to do it as a group, not just as individuals. In this phase, the leader has the most time to himself, and usually he can write more code that before and he has to be careful that the team does not slip into a day-by-day mode due to an increasing rise in requirements that come from the client or management.

From my personal experience I have seen only two teams that can be truly called self-organizing (the second team I lead and another one in the same company). I interacted with three teams that were in to Evolution Stage (the first one I led and two others in the same company) and about two in survival stage (the third one I lead and other seven in the two companies I've been employed). This numbers can not be representative for the whole industry because they are just my personal experience that is not validated by any study or multiple peers.