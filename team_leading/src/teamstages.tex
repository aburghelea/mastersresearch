% vim: set tw=78 sts=2 sw=2 ts=8 aw et ai:

Bruce Tuckman prosed in 1965 a team development model based on 4 stages and evaluated based on the relationship between the members. The stages of the Tuckman model are:
\begin{description}
\item[Forming] when the team is highly dependent on the leaders direction, they do not know their roles and there is no relationship between the members.
\item[Storming] when the members can't see eye to eye and conflicts are often because each one wants his idea to be used in the detriment of the other ideas
\item[Norming] when relationships between the members start to develop and consensus largely forms inside the team.
\item[Performing] when the team has a clear strategy and knows what to do and the team leader delegates and oversees the actions.
\end{description}

After going through these stages the team can be seen as it interacts and develops with three kind of interactions:
\begin{description}
\item [Day by Day] It survives day-by-day doing their tasks and they do not have any time free to learn new things
\item [Evolution] The team has time to learn new things in parallel with developing the new features
\item [Self-Organization] Each member of the team is capable to balance his tasks so that he is able to learn new skills and he chooses what to learn by himself
\end{description}

This phases are also depicted by Roy Osherove \cite{notes-to-a-software-team-leader} with the names: Survival, Learning, Self-Organizing

\subsection{Day by day stage}

For a team to be in a day-by-day stage an important requisite must not be present. This requisite is the time to learn. The team is focused only to deliver the tasks on their plate and the workload is big enough that they do not have any time to improve their personal skills. Although this could seem that the perfect stage from a management point of view and interpreted as "the team is working full-time, using the maximum amount of time to be productive" this can lead to an unmotivated team that in time will lose it's productivity and it will also lead to a high turn-over in the company. 

The leader of such a team has the duty to create slack time. It is mandatory for the management and client to understand the the current working speed will lost for a maximum of 2-3 months until the team members become frustrated and take into account leaving the company.

The leader must obtain the buy-in from the client and the management to create slack time. This slack time can be created by removing a part of the current commitments. In agile this is called a sprint scope change.

Although nobody will want to push the deadlines, experienced managers will understand that a demotivated team will not ensure delivery on time. Experienced managers agree that the key to success is a motivated team that is always helped to learn and develop new skills.

\subsection{Evolution stage}
\todo{when the team learns new skills}

\subsection{Self-Organizing stage}
\todo{when the team know what to do}

\todo{chapter closing lines}