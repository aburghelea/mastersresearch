% vim: set tw=78 sts=2 sw=2 ts=8 aw et ai:

The career path of a programmer is a complicated and difficult one. The start is as straight forward as it gets. You start as as a junior and learn from the people in your team, you make a lot of mistakes, the same mistakes that others have makes and usually you come up with new ones. After this you gain some experience and increase your knowledge base, ultimately becoming a senior. Sometimes it ends there, you have a lot of technical knowledge and you are referenced as a \textit{coding guru}, but sometimes you it things take an interesting turn: \textbf{``You end up in charge of a team!''}. And if you are not a natural born leader you panic and end up with three questions:

\begin{enumerate}
	\item What really means to be a team leader?
	\item Why me?
	\item How am i going to pull this off?
\end{enumerate}

Before you become productive in the new position, you have to learn what the answers to this questions really mean to you.

\subsection{Team Leadership Concept}
\label{sub-sec:teamleadershipconcept}
\todo{What is team leadership}

The agile movement, especially the Scrum Methodology, has created a lot of fuss and on average 7 out of 10 companies try to organize their teams in this manner. Based on the Scrum Methodology there are three roles \cite{scrum} inside a scrum team:

\begin{description}
	\item[Product owner] that holds the vision for the product
	\item[Scrum Master] that helps the team best use Scrum
	\item[Development team] that builds the product
\end{description}

Except the \textit{{Scrum Master}} role, there is no other role to hint that there is a dedicated person to coordinate the team. This fact and the 11\textsuperscript{th} Agile Principle \textit{``The best architectures, requirements, and designs emerge from self-organizing teams''}\cite{agile-manifesto} hint have created the common misconception that there is no need (and even no place) for a team/technical leader. This misconception could not be further that the truth. Philip Anderson event refutes this assumption declaring:

\begin{displayquote}
Self-organization does not mean that workers instead of managers engineer an organization design. It does not mean letting people do whatever they want to do. It means that management commits to guiding the evolution of behaviours that emerge from the interaction of independent agents instead of specifying in advance what effective behaviour is.
\end{displayquote}

But in the simplest way, team leadership is the art and science of being creative an productive through other people on the long run, regardless of the steps one takes to achive this.


\subsection{Motivation}
\label{sub-sec:motivation}
\todo{The impediments i had as a team leader in coordinating AFI and CR}

\subsection{Objectives}
\label{sub-sec:objectives}
\todo{RD: In general e indicat sa fie masurabile obiectivele. Sa ai un fel de valoare numerica care poate cuantifica cat de bine e indeplinit un obiectiv.}

\todo{Decisions made by me in coordinating CR and possible improvements}


\todo{Small synthesis of chapter}
