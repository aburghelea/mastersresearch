% vim: set tw=78 sts=2 sw=2 ts=8 aw et ai:

The career path of a programmer is a complicated and difficult one. The beginning is very lean and easy to handle. Starting as a junior and learning from the other team members, natural mistakes are usually made. This mistakes are often accompanied by some personalized mistakes. With time, experience is gathered and the knowledge base increases and the seniority level increases. While in some cases the path continues on the technical side, the person in cause becoming a \textit{coding guru}, there are some cases when the path forks to a branch that needs a lot of people skills, in the form of team leadership. While it may be natural to coordinate a team, most of the new team leaders start this process with 3 questions:

\begin{enumerate}
	\item What really means to be a team leader?
	\item Why me?
	\item How am i going to pull this off?
\end{enumerate}



Before becoming productive in the new position, the team leader has to define how he identifies with the questions above and what the answers mean to him.

\subsection{Team Leadership Concept}
\label{sub-sec:teamleadershipconcept}

Team leadership is a concept that has it's core values the coaching of the members in order to deliver high quality products on time. While team leadership comes in many flavours in the software industry, the key questions a team leader should handle follow the same topic. The objective of this report is to define a common list of indicators a leader can follow in order to achieve a self organizing team.

The agile movement, especially the Scrum Methodology, has created a lot of fuss and on average 88 out of 100 companies try to organize their teams in this manner\cite{agile-survey}. Based on the Scrum Methodology there are three roles \cite{scrum} inside a scrum team:

\begin{description}
	\item[Product owner] that holds the vision for the product
	\item[Scrum Master] that helps the team best use Scrum
	\item[Development team] that builds the product
\end{description}

Except the \textit{{Scrum Master}} role, there is no other role to hint that there is a dedicated person to coordinate the team. This fact and the 11\textsuperscript{th} Agile Principle \textit{``The best architectures, requirements, and designs emerge from self-organizing teams''}\cite{agile-manifesto} hint have created the common misconception that there is no need (and even no place) for a team/technical leader. This misconception could not be further that the truth. Philip Anderson event refutes this assumption declaring:

\begin{displayquote}
Self-organization does not mean that workers instead of managers engineer an organization design. It does not mean letting people do whatever they want to do. It means that management commits to guiding the evolution of behaviours that emerge from the interaction of independent agents instead of specifying in advance what effective behaviour is.
\end{displayquote}

But in the simplest way, team leadership is the art and science of being creative an productive through other people on the long run, regardless of the steps one takes to achive this.


\subsection{Motivation}
\label{sub-sec:motivation}

From June 2013 and until January 2015 I've been involved in coordinating 3 teams with a pause of about 7 months when I was just developing. The situations I encountered as a team leader are:

\begin{enumerate}
\item 8 junior team members coordinated by two senior members
\item 2 senior team members (including myself)
\item 7 team members with mixed seniority levels while two of the members had more technical expertise than I had.
\end{enumerate}

Coordinating a team formed from junior team members is very easy because they will look up to you and they will follow your decisions based on the notion that the leader has more experience in the industry and they will strive to learn instead of challenging the decisions. From personal experience I had observed that the junior members will not challenge technical decision and they simplified my job because I could focus more on the product and not on debating. Coaching was not there to stimulate debates because the members had no prior experience to generate different opinions.

A small size team will also be easy to coordinate if the project is big enough to ensure that the members work on different modules where they have full ownership. This will generate only mild debates on best practices and integration issues, but in most cases, every developer will be in charge of making development decisions. Personally I consider that the success of my second leadership attempt was dues to both a small team of senior developers and a good relationship with the client based on trust.

A big team with mixed seniority level will be a more challenging to coordinate because technical difference will often occur and mediation will be necessary. My third experience with leading a team was marked by tendency to challenge every decision just for the sake of it. Gaining their trust was one of the hardest things I had to do in my professional life, but unfortunately it came a little too late. 

The motivation behind this report is to give an incentive over what team leadership means and what new team leaders should expect.

\subsection{Objectives}
\label{sub-sec:objectives}

The objectives of this study is to validate that leadership is a complicated science close to the rank of art and provide a set of ten questions that a leaders should ask himself every time to validate that his leadership approach is valid. 
Along side with the questioner some personal decisions in leading the teams were valid.