% vim: set tw=78 sts=2 sw=2 ts=8 aw et ai:

The career path of a programmer is a complicated and difficult one. The start is as straight forward as it gets. You start as as a junior and learn from the people in your team, you make a lot of mistakes, the same mistakes that others have makes and usually you come up with new ones. After this you gain some experience and increase your knowledge base, ultimately becoming a senior. Sometimes it ends there, you have a lot of technical knowledge and you are referenced as a \textit{coding guru}, but sometimes you it things take an interesting turn: \textbf{``You end up in charge of a team!''}. And if you are not a natural born leader you panic and end up with three questions:

\begin{enumerate}
	\item What really means to be a team leader?
	\item Why me?
	\item How am i going to pull this off?
\end{enumerate}

Before you become productive in the new position, you have to learn what the answers to this questions really mean to you.

\subsection{Team Leadership Concept}
\label{sub-sec:teamleadershipconcept}

The agile movement, especially the Scrum Methodology, has created a lot of fuss and on average 88 out of 100 companies try to organize their teams in this manner\cite{agile-survey}. Based on the Scrum Methodology there are three roles \cite{scrum} inside a scrum team:

\begin{description}
	\item[Product owner] that holds the vision for the product
	\item[Scrum Master] that helps the team best use Scrum
	\item[Development team] that builds the product
\end{description}

Except the \textit{{Scrum Master}} role, there is no other role to hint that there is a dedicated person to coordinate the team. This fact and the 11\textsuperscript{th} Agile Principle \textit{``The best architectures, requirements, and designs emerge from self-organizing teams''}\cite{agile-manifesto} hint have created the common misconception that there is no need (and even no place) for a team/technical leader. This misconception could not be further that the truth. Philip Anderson event refutes this assumption declaring:

\begin{displayquote}
Self-organization does not mean that workers instead of managers engineer an organization design. It does not mean letting people do whatever they want to do. It means that management commits to guiding the evolution of behaviours that emerge from the interaction of independent agents instead of specifying in advance what effective behaviour is.
\end{displayquote}

But in the simplest way, team leadership is the art and science of being creative an productive through other people on the long run, regardless of the steps one takes to achive this.


\subsection{Motivation}
\label{sub-sec:motivation}

From June 2013 and until January 2015 I've been involved in coordinating 3 teams with a pause of about 7 months when I was just developing. The situations I encountered as a team leader are:

\begin{enumerate}
\item 8 junior team members coordinated by two senior members
\item 2 senior team members (including myself)
\item 7 team members with mixed seniority levels while two of the members had more technical expertise than I had.
\end{enumerate}

I consider that the first two projects were very successful while the last one was a bitter fail. The paradox is that for the first two teams I did not plan or even think of what I want to achieve from a team perspective, I just new what the product should do when it's complete. For the first team I had the advantage of being on the same page with the other seniors and together we were more experienced from a technical point of view that the other members. What we said ... went. We had no technical challenges rose from the team, the team was there to learn and we were the ones that had the answers. The second team was the easiest team to lead. Before I was appointed as an informal (not on paper) team leader, we had already formed good relationships with the client. He trusted us and we had free hand as long as the end result behaved accordingly. We both new what we have to do and were organizing ourself as we saw best to deliver on time. As a team leader, I did not have to do anything to ensure a team cohesion or to deliver. This both situations have in common the fact that I never took the time to analyse what I expect from the team or what the team  expects from me. I rushed in to lead the team not knowing what a team lead should do. The success of both these teams stood in a contest of circumstances and even a big dose of luck and almost not at all on my decisions and guidance.

The last team was the most difficult one to lead. My technical skills were very poor compared with the two other members. Their technical opinion mattered a lot but they did not want to directly lead the team. In the same time they had a tendency to challenge every decision just for the sake of it. Gaining their trust was one of the hardest things I had to do in my professional life, but unfortunately it came a little too late. Although this was the first time I tried to analyse and plan my actions as a team leader, but still I did not think to read any books and leverage the expertise that others have in the field.

The motivation behind this report is to give an incentive over what team leadership means and what new team leaders should expect.

\subsection{Objectives}
\label{sub-sec:objectives}

The objectives of this study is to validate that leadership is a complicated science close to the rank of art and provide a set of ten questions that a leaders should ask himself every time to validate that his leadership approach is valid. 
Along side with the questioner some personal decisions in leading the teams were valid.