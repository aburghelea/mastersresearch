% vim: set tw=78 sts=2 sw=2 ts=8 aw et ai:

Software companies have different ways to organize their teams, each based on the development methodology they use, or better said based on their interpretation of the methodology.
\subsection{Existing visions about team leaders}
\label{sub-sec:existingvisionsaboutteamleaders}

There are cases when the leadership of the team is usually done by the person who has the most technical experience, sometimes called as a \textbf{coding guru}. The reason why this kind of person is appointed as team leader is the fact that  the senior management needs a solid technical voice to represent the team in front of the client. On the other hand the most technical person may not be oriented on developing the skills of the team members and is usually focused only on delivering the project (often by a great effort on the coding part from his side)

The last kind of team leader is the person that can communicate easily with the team members and also with the clients or senior management. The reason a person with great soft -skills is in charge of a technical team is represented by the fact that he can usually ask the correct questions in order to help each member solve the issue that he is working on, while not providing the correct answer. This fact helps the team to grow by empowering the members and by helping them reach the answers they need.

The most popular development framework currently used, Agile Scrum, usually has a mediator, and not a leader called Scrum Master. The Scrum Master has the main responsibility  to solve impediments of non-technical (e.g.: obtaining hardware resources, facilitating meetings). Not being mandatory for the Scrum Master to be a technical person, he does not have the obligation to coach people, but he must ensure that the team is not impended because of lack of resources.

\subsubsection{The person with the most technical experience}

During the periods when projects were being developed using a \textit{Waterfall approach} the team leader position was filled by a person with a lot of experience in the industry (usually more than 7 years). His main objective (set by the company management) was to deliver the project on time with the team that he has. The main advantage of this kind of person would be his vast technical knowledge that would help him design the architecture of the application and enforce decisions of how the application is implemented. Usually this kind of leader is called a \textbf{technical leader}.

The main skills that are taken into account when a person is appointed as a technical leader are:

\begin{itemize}
\item has a vast experience with the kind of project that is being developed
\item has good knowledge of programming languages, APIs and frameworks
\item understands the impact of changing the frameworks or APIs
\item has the ability to make technical decisions
\item is good at explaining technical solutions
\item is accountable for his decisions
\item is loyal to the company and management team
\item is trustworthy
\item gets things done
\end{itemize}

The disadvantages of having a technical leader is represented by the fact that he alone has the objective and power to steer the design of the application. It is his duty (and his alone) to decide how the architecture will be implemented and what technologies will be used. This approach does not empower the team to develop their skills and thus the members will not come with new ideas or possible technical approaches. Often the team members see themselves as implementers of the ideas guided out by the leader. 

While the technical leader delegates the implementation to various team members he will often have the tendency to assign the implementation of each module based on the seniority level of the member that will to implement it. He will assign a task to a member that has all the skills to solve it in time. The assigning decision is made only on the previous experience of the member and a technical leader will not challenge the members by giving them tasks that they can not solve with their current skills. By doing this, he does not develop the technical skills of the team. Using this approach, the technical leader will always implement the most difficult tasks by himself believing that he alone can solve them. This requires extra effort from his part to balance the design of the application, implementing his own technical tasks and providing fast technical answers to the team when they get impeded.

From a team perspective, the technical leader has to have the following traits:

\begin{itemize}
\item is fair
\item gets along with the members
\item is available and approachable to provide strait forward answers to technical questions
\item has good understanding of the business domain
\end{itemize}


\subsubsection{The person with great soft-skills}

The second flavour of team leader is the person who know how to ask the right questions in order to help his peers reach the answer by themselves, thus developing their problem solving skills without providing the solution.

Roy Osherove describes this person as a leader that coordinates his team by the following quote \cite{notes-to-a-software-team-leader}:

\begin{displayquote}
``There are no experts. There is only us''
\end{displayquote} 

The reasoning behind the previous quote is not the fact that the team has to develop a product based just on their current skills. It means that the team should not wait for the arrival of a person that already has experience to use the necessary tools and frameworks, but instead they should become the experts by developing their skills and knowledge base. 

This approach has a mandatory requirement of slack time. Slack time does not mean that the team does not work for the company and wastes time browsing the internet. Slack time represents the period of time necessary for the team to learn how to use various technologies. It is called slack time because the output is not directly visible in a set of finished tasks for the project functionality. The output of slack time is a set of new skills that the members of the team have acquired by learning and experimenting with new tools.

The main objective of this kind of team leader is to motivate and coordinate a team in order to deliver good software with the effort of the team. The team leader is not necessarily the most experienced programmer from the team, but is able to clearly communicate the project expectations. He has to create enough slack time to ensure that the team has the necessary time to learn.

The main skills that are taken into account when a person is appointed as a technical leader are:

\begin{itemize}
\item has good communication skills
\item has people management skills
\item can motivate the team members to learn new technologies
\item can steer the development focus
\item know how to manage time
\item understands that the team knowledge is more important that personal experience
\item is able to delegate work
\item is a good listener, respects other people's opinions
\item is a good conflict resolver
\end{itemize}

The difference between a team leader and a technical leader is that the team leader does not make the technical decisions by himself, he challenges the team to come up with solutions and he just signs of on the agreed one, while the later decides how things will be implemented and delegates just the execution. The team leader encourages the team to develop their decision making skills by empowering the team, making them accountable for the decision; The team decides what and how it is implemented, the result is their work from design phase to implementation. This approach ensures a better engagement from the members because they are not just implementers, they are influencers, their ideas impact the project directly.

The team leaders responsibility is coaching. This means that he has the role of helping the team make decisions instead of making them for the team. The main weapon of a team leader is empowerment. He lets the team steer the project and helps them solve differences by asking questions, not answering.

\subsubsection{Scrum Master}
The third flavour of leader is the Scrum Master. In comparison to the technical and team leader the Scrum Master does not necessarily need to have a technical background (although it is not uncommon for the Scrum Master to be a former programmer or to be part of the development team). The Scrum Master is more of a process orchestrator.

The Scrum Master has two main responsibilities:

\begin{itemize}
\item guiding the team in adopting the Scrum Process
\item resolving impediments like lack of resources, lack of business details
\end{itemize}

A Scrum Master will observe the team processes and will provide improvements to the processes that prove unproductive. He will ensure that the workload for the development period (sprint) is within the team capacity (velocity). He does that by tracking the number of story points that the team delivered in the current sprint and ensures that the team does not commit more work that an average velocity.

The Scrum Master does not provide technical support, but ensures that the team communicates constantly and facilitate meetings. By taking this onto himself he ensures that the team is not interrupted from the technical tasks.
\begin{table}
	\begin{center}
	\begin{tabular}{| p{4cm} | p{4cm} | p{4cm} | p{4cm}|}
		\hline
		 & Technical Leader & Team Leader & Scrum Master \\ \hline
	    Technical Knowledge & Very strong & Medium to advanced & Not a requirement \\ \hline
	    Main Objective & Making the technical decisions & Coaching the members & Facilitating Scrum process \\ \hline
    	Management objectives & Business Know how and accountability & People skills and conflict resolver & Non technical impediments resolver \\ \hline
	    Most suited leadership approach & Leader-Follower & Leader-Leader & Facilitator \\ 
\hline

		\end{tabular}
		\caption{Comparison between the leadership types}
		\label{tab:leader-comp}
	\end{center}
\end{table}
\subsection{Leader-Follower model vs Leader-Leader model}

In team coordination the actions of the lead can be classified into two categories as follow:

\begin{description}
\item [Leader-Follower] - where the person in charge is responsible for all the decisions, while the team is responsible for implementing the leaders' decision
\item [Leader-Leader] - where the person in charge empowers the team to make decisions, while he is more of a coach than decision maker.
\end{description}

The \textit{Leader-Follower} has his origin in the military field where the commanding officer gives an order and that order is cascaded down the chain of command to the soldiers that have the obligation to execute it without hesitation. 

The U.S. Navy has a famous quote depicting the \textit{Leader-Follower} model:

\begin{displayquote}
Leadership is the art, science, or gift by which a person is enabled and privileged to direct the thought, plans, and actions of others in such manner as to obtain and command their obedience, their confidence, their respect and the loyal cooperation. \cite{fundamentals-of-naval-leadership}
\end{displayquote}

In other words this leadership model is about controlling people. While in the military this approach is mandatory to ensure a well oiled fighting unit, in the the software industry this model fails.IT professionals want to use their intelligence not just to implement features, they also want to influence the way features are implemented.

The \textit{Leader-Leader} model is totally different from the \textit{Leader-Follower} model by the fact that the official leader from such a structure is obliged to ensure that he is not a single point of failure.

Leadership is not some mystical quality that some possess and other do not.\cite{turn-that-ship-around}

This model provides great improvements to both morale and effectiveness and it also makes the team stronger by providing three key features:
\begin{enumerate}
\item The official leader is not a single point of failure and can be easily replaced in case of urgency (eg: unscheduled absence, the leader leaves the company) by empowering the team to decide by themselves
\item The members have a better morale by being allowed to decide what and how features are implementing, this proving that they are not seen just as \textit{coding monkeys} but they are decisions makers
\item There are no dead times while waiting for the lead to decide before they can implement.
\end{enumerate}

The success of the leader using a \textit{leader-leader} was greatly emphasized by Andrei Piti\cb{s}:

\begin{displayquote}
A leader can be promoted only when someone he managed can take his place.
\end{displayquote}

If a leader has managed to develop the skills of his direct report enough so that they can decide by themselves without his intervention, he has managed to create a self-organizing team capable of solving their problems and impediments.

Although in theory the \textit{Leader-Leader} model seems revolutionary and desired, in practice it has been observed that people will fear to make decisions if their position does not require it, thus being hard to implement it first time. On the other hand the \textit{Leader-Follower} model may seem despotic, but it is easier for a team to work in this manner, because they do not have to leave their comfort zone, but it can easily reach a stage where the members are bored by the lack of challenges.

Before you chose what kind of leader you want to be and what kind of model to use, take into account Cockburn's cite:

\begin{displayquote}
If you do not know anything about human behaviour, you know very little about software development.
\end{displayquote}

Chose the right model based on the project constraints and the long term objectives.