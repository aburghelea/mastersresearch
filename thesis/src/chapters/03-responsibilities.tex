\chapter{Expectations and Focus Areas}
\label{chapter:responsibilities}
Each team is expected to deliver products at a company accepted pace and usually this falls on the shoulders of the leaders. Having a clear organizational hierarchy or matrix, intermediate level leaders are expected to represent the teams they coordinate in front of clients and showcase their success. In the same time success is measured differently by each person and situations.  To avoid situations when the what a team consider success, management consider failure expectations should be communicated and agreed in advance.
\section{Expectations from Leaders}
\label{sec:fromleads}
Leaders being in the middle of the team and the upper management, expectations from them come from two directions. 
Team leaders are in charge of building teams and leading them towards success, while people managers are in charge of developing each individual on the long run. 

From the upper management point of view, leaders are expected to give direction to the team in order to coordinate with the company values, missions and objectives. In the same time they are expected to set up a plan for monitoring and evaluating the progress towards the team goals. In the same time leaders are expected to be team players and not individual developers.

From a \textbf{give to the team} point of view, expectations from leaders are clear, but usually the \textbf{take for the team} point of view is not always clears.  But usually the team expects the leaders to solve all their impediments. This fact is usually misinterpreted as it should not mean that the leader should do the work that the team can not, but rather to help them in solving the problems that arise. In some situations it may mean that a leader has to obtain information from other involved parties, but usually a good leader will point his reports in the direction that will help them solve their problems. In the same time, leaders are expected to give credit and make the team success visible to the stakeholders while in the same time to take responsibility first to any failures.

People managers being in charge of personal development of the company employees, thus they should be example setters, modeling the expected behaviors. In the same time people managers are expected to be fair in evaluation and proactive in solving impediments that his reports have.

Regardless of the roles, leaders are expected to plan and execute for the entire duration of the project in such a way that the team is both motivated and engaged in the development of the products.

\section{Expectations that Leaders Have}
\label{sec:byleaders}
In the same that team members have expectations from leaders, the reverse is still true. In order to lead a successful team, the members need to be on the same page with the leaders. For this to happen every member (including the leader) must be mature enough and be able to talk things out, raising problems as soon as they arise. It may be the responsibility of the leader to guide the team in overcoming difficult situations, but it is the responsibility of everybody to signal possible impediments.

In the same time leaders expects from his team integrity and not waste time instead of working. Although it may sometimes be the case that technical tasks are not to the liking of the members, but they can should not refuse to work. 

From my opinion the most important two expectations that leaders have is for members to continually improve their skills and also be team players in order to create a long lasting and self organizing technical team.

It is easy to understand that a team member is expected to pull his own wight but also displays genuine commitment, giving more than their full potential but also going the extra mile. Giving more than a person is asked is what makes the difference between a good team player and a great one. Coming up with creative ideas and adapting quickly to the moving project requirements create a truly reliable employee, adding value to both the team and the company.

Knowing how to be flexible and not fight change represent a new opportunity for growth , while supporting other people by both offering feedback and help when needed showing willingness to collaborate makes a good impression on both managers and team.

Most of the candidates (9 out of 14) described the perfect team member as being easy-going persons, focused on both project and personal development that are willing to do more than what is stated in their job description, but also help the leaders in guiding, controlling and implementing new procedures that will better help the team in achieving their end goal.

From a top level, leaders expect their colleagues to be cooperative, flexible and driven to success that use their full potential to prove both themselves and also help the company reach it's own goal. In the same time, it is important to note that a team sometimes needs doers(sometimes called mercenaries) and not just visionary, every team needs persons that can focus for long periods of time to get things done because a team full of ideas but with no executors is just a team of dreamers and never of team of deliverers.
\section{Focus Areas of Leaders}
\label{sec:focus}
\fig[scale=0.7]{src/img/focus.png}{img:focus}{Focus areas of leaders}
Each type of leadership position has it's own advantages and disadvantages.
From \labelindexref{Figure}{img:focus} it can be observed that the focus areas shifts from project development to personal developments while leaders transition from technical to mentoring and in the end, managing.

It is observable that technical leaders, as their role name states focus primarily on the solution of the project and not the team itself. The majority of the technical leaders try to find the best candidate from the team that will be able to deliver the best results for the active tasks. While others focus on checking up on how things are going. Although not it is a good thing to supervise the progress so that if intervention is needed it can be done as soon as possible. Compared to \labelindexref{Figure}{img:lifespan} and the fact that technical leaders that have coordinated the same team for more than six months consider that they act more as coaches, the fact that their main focus is still the project and not the team proves that technical leaders are still more of technical result oriented people, rather than focusing on building durable teams.

Keeping in mind that people managers do not work on the same project team as his direct reports, their main focus becomes the development path of each employee for a longer period of time. Projects always have a start and end date, after which an employee will be assigned to a new project, most possibly with a new team. In order to ensure that the employee will be ready to take on new projects (also meaning new requirements, technologies and colleagues), he must be able to quickly adapt. In an industry where new languages appear very often and where technologies change, it may be a mistake for an software engineer to focus only on one single technical path. The risk of that path becoming obsolete from an industry point of view by the time that a person becomes an expert on it has become more and more visible.
\fig[scale=0.7]{src/img/failure.png}{img:failure}{Biggest failure}

From an focus are point of view, team coaches prove that they are have an approach in between technical leaders and people managers. They tend to focus on building project teams for the long run, most of them having as goals to take new projects with the same team that they have worked in the past. By this the show that they focus more on the people and not on the project itself.

It is interesting the fact that most of the leaders see the failure of the team as their failure, \labelindexref{Figure}{img:failure}, showing that leaders assume the fact that they are responsible for their team. It is important to note that the 3 candidates that considered the fact that they did not form for a great team were team coaches. From the point of view of biggest successes, \labelindexref{Figure}{img:success}, the situation changes and most of the candidates still consider personal successes bigger than the success of the team. The two candidates that considered the team performance a success were people managers.

\fig[scale=0.7]{src/img/success.png}{img:success}{Biggest success}
Upper management expects leaders to deliver projects on time and help people grow in order to accommodate the future plans of the company and in return leaders expect their superiors to provide accurate goals, desires and direction. In the same time leaders want from their reports to pull towards the same goals and have a self-driven attitude, while they are ready to offer guidance and technical support in order to achieve success. Although leaders consider themselves to blame when thing go wrong with the team, they should also let go of their own egoism and be cheerful when the team succeeds, keeping in mind that he/she is the person that set the trend and offered guidance. 