\chapter{Conclusion and Future Work}
\label{sec:con-fut}

\section{Conclusions}
Leadership, regardless of the assigned role, even if we are talking of about team leaders or people managers, represents the ability to use the power of other to deliver the tasks that you are assigned. 

It has been observed that although the technical leaders focus mainly on delivering the products on time while they are the driving the entire technical direction, the enforcement approach becomes toxic if the team stays in place for more than 6 months. In the same time people managers are focused more on developing each individual for the direction in which the company wants to grow. Team coaches have demonstrated that they have a hybrid approach by mentoring the team in the desired direction but also giving a free hand to his reports. While technical leader try to focus more on tasks of the project, people managers focus more on what would be useful in the future to have in the knowledge base of the company.

Similar to the focus areas, pure technical leaders prefer to enforce their decisions while team coaches and people managers prefer to motivate and only if mandatory to enforce.

When it comes to hiring, salary adjustments and firing, team leaders are in a position of consulting while people managers and upper management have the requirement to make the decisions. On the other hand delegation and toxic employee management are the responsibilities of both kind of leaders.

Age and number of years spend in the industry prove to always be a barrier in leader acceptance, but with building trust across the team and the leader proving his capabilities it can be overcome. On the other hand, ego and uncorrected problems are much more difficult to overcome often being necessary for outside intervention.

Trying to lead a team with hidden motives and without full honesty will always lead to different results than expected. Without being completely open, fair and also firm, the expected result will also be different than expected. On the other hand, focusing more on how the leader can be better and rather than correcting the direct problems that he faces without finding the root causes will render better results, during the desired time-line and also with having a better formed and motivated team.

Leadership is both knowledge and art, it can not be learned as an algorithm can, but by analyzing each member, how he thinks, what he likes and how he reacts, combined with previous experiences will lead to coordinating a good team (both technical and organized) that will be able to add value to the company.

Further development of this study can be conducted in experimenting complete delegation of technical decisions inside teams formed from junior or mid-level engineers. Another approach to be studied would be the implementation of a periodic 360\textdegree feedback where the entire team would gather in a meeting room or outside of the office, by turn each individual would  write on a piece of paper what he likes, what he dislikes and what should be improved for each of his colleagues. Afterwards each member will read all the notes written to him and sum up all the parts and propose an action plan to improve. In the next meeting everybody will state what has improved and what still needs to be improved.

Neither leadership approach is perfect, but for long lasting teams a dual approach of both team coaching and people management will create a safe, pleasant environment and performing company.