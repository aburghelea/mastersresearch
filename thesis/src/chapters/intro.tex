\chapter{Introduction}
\label{chapter:intro}
In any industry, there is no one man team. Although there are situations where for amusement purposes a single person does in parallel all the activities of an entire/team band, the scope is always the same: entertainment. But to obtain performance, a team is necessary. The situation can be easily compared to the programming concepts of threads: you have more workers that do separate tasks, but they need to be somehow synchronized or coordinated. Teams are in theory the easiest way to increase the throughput, but where there are multiple persons there also are different opinions. Although the boss was the person in charge of controlling and setting the direction, the position now shifts to leadership where control is relinquished and direction is hinted.

Leadership is present in most of the possible environments and industries, but the one that I find most interesting is the IT industry. The reason behind the interest is the fact that the IT industry is considered a spoiled industry. A spoiled industry is described as being an industry where the average salary packages are above (at least 3 to 4 times) the average salary in a country and where the number of open positions is higher than the number of possible candidates. Two companies that I have interacted with (names undisclosed) reported that the number of accepted employment offers is between 50-60\% of the total of extended offers. 

In the situation when it is hard to find new employees it becomes mandatory to have a high retention rate. Due to this fact, IT companies have started to use people management techniques in order to maintain the existing employees happy and performing.

\section{Team leader}
\label{sec:tl}
\cite{abur-tl}
\section{People management}
\label{sec:proj-scope}
\cite{abur-pm}
\section{Motivation}
\label{sec:motivation}

\section{Objectives}
\label{sec:Objectives}
