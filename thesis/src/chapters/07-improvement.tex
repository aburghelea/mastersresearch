\chapter{Improvement Experiments}
\label{chapter:improvement}
Neither leadership role is perfect for each situations. Both team leaders and people managers are necessary roles: team leaders are always needed, while the role of people managers is seen in appreciated in big companies.

At a basic level, a leader is a person who leads others. Leaders are persons who drive others, have a vision and they determined to achieve their vision. In the same time leaders do not have to be officially appointed in order to motivate and inspire others.

Based on previous experience in leadership positions 3 experiments have been carried out in order to observe their efficiency and possible uses. The first two experiments (Unofficial coaching, Behavioral coaching) were carried out without having an official leadership position, while the last one (Technical dominance) was carried out from an official technical leadership role.

\section{Unofficial Coaching}
\label{sec:un-coach}

The experiment was done in a context where the person being coached was a junior-middle level software engineer (based on knowledge), one year older than the coach, being in separate project team. The coach had only a software engineer role in the company, but had a 2 year previous experience in leadership. Except the coach and upper management no one was aware of the experiment being conducted.

The person being coached was manifesting an immature behavior and was always unreliable, coming into office after noon (while the expected arrival time was 9:30-10:30 AM). Coming into office so late and being involved in many projects, meant that the team was not able to gather information about the progress he made, resulting this in unnecessary delays. 

The experiment in cause was to try and solve the situations by mentoring the nonperforming employee by a colleague of the same age, but working in a different project team.

The steps of the experiments were:

\begin{enumerate}
\item Establish a personal relationship by taking smoking breaks at the same time and using the time to discuss the progress of the college classes (common point of interest outside of work) and also project development.
\item Finding out the employees in cause of the next deadlines and if they will be meat. His feelings was that there is no problem in delivering on time
\item\label{uc:st} Presenting technical coaching by recommending documentation, tutorials and giving ideas of what frameworks 
\item Comparing the employees view of the project time-line with the real one and asking his view on the discrepancies.
\item Giving firm feedback on his performance, using the \textbf{F}acts-\textbf{A}ctions-\textbf{C}orrections model \cite{abur-pm}
\item Guiding the employee to come up with a plan of actions in order to recuperate from the unpleasant situation. (Arrive in the office between 9:30-10:30 AM, Respect deadlines or announce in advance that there are impediments that will cause delays
\item Ask questions similar to \textbf{why} in order to confirm that he has understood why the action plan is important and what are the expected results.
\item\label{uc:end} Guide the employee into using a commitment language \cite{notes-to-a-software-team-leader} in order to inspire and establish engagement with the action plan.
\item Monitor the progress and appliance of the action plan.
\end{enumerate}

Unfortunately, the employee considered that his personal life was more important and that it was acceptable to go to sleep at 5:00 AM during working days and that it still is no problem that he is not available for his team to provide information about his tasks. After iterating steps \ref{uc:st}-\ref{uc:end} over a period of 2 months with this kind of discussions occurring once every two weeks.

After two months with no improvement the decision to terminate the employee was made. When asked why he did not take the recommendations into account a nonacceptance of age ( \labelindexref{Section}{sec:unnaceptance}) barrier was discovered, considering that the unofficial coach did not have enough experience in the industry, thus preferring to follow his own intuition. In the same time, being unaware of the repercussions in case the situation would not improve the employee considered they are none.

Trying to motivate employees by coaching them without a clearly presented motive and presenting only the possible gains but not the repercussions does not have the necessary leverage to improve the situation. Andrei Piti\cb{s} declared that it is important to \textit{"Do what you say and say what you do!"}. Without being open of the situations and trying to improve it from the shadow, employees do not see the real impact of his behavior and possible unpleasant results, the experiment failed thus not being a real possible improvement in leadership without improving the desire plan of action.

\section{Behavioral Coaching}
\label{sec:beh-coach}

While the previous experiment \ref{sec:un-coach} has an approach of offering unsolicited advice, the second experiment was conducted with a new team leader, in charge of a team newly formed. The experiment was done in the context where the new leader approached the coach, who had previous experience in both people management and team leadership, for advice on overcoming day to day challenges created by the team. 

The new leader had to overcome three major problems:
\begin{enumerate}
\item Taking ownership of the existing project that was poorly written and the next deadlines were not negotiable and immediate
\item Managing an old slacker employee
\item Managing new team members with attitudes suited for corporations and not start-ups.
\end{enumerate}

Seeing that the situation of the project and deadlines hint a technical leader approach, the fact that more that 50\% of the team members were new to the company, and most of them coming from outsourcing corporations, where the accent is put more on following procedures rather than making decisions on the spot, a team coach approach was suggested.

During the one-on-one meetings the new team leader would present a recent situation (e.g.: employee John Doe started to work on something else that the tasks that were mandatory; employee Jane Doe prolongs the meetings in order to debate the faulty processes instead on focusing on the immediate delivery of the project; the team refuses to work on the delegated tasks).

Upon observing the interactions between the leader and his team it was observed that:
\begin{enumerate}
\item The leader had created a much to friendly environment, determining the team to focus more on personal interactions rather than the task.
\item The attitude of the leader was soft enough, using expressions that would induce the atmosphere of: \textit{You would do me a favor by doing this, but it's OK if you do not help me}.
\item The leader was working extra hours in order to pick up the load that the team was not filling, without making it visible that he is doing extra work that could be easily accommodated by them.
\item The leader did not communicate his expectations and the objectives of the project.
\end{enumerate}

After the new leader presented in detail a couple of situations where he did not obtain the results that he had hoped. He was asked to formulate the situations in the following format:
\begin{enumerate}
\item Describe the context of how the tasks were assigned.
\item Describe the person that the task was assigned using the PRAE model
\item Evaluate the language of attitude of the leader based on the usual traits of the PRAE personality
\item Identify the differences between the used approach and the PRAE recommended approach.
\item\label{fm:example} Provide examples of how the mentor would have approached the situation.
\end{enumerate}

The examples mentioned at \ref{fm:example} were formulated to be open, stating a personal opinion and not a standard:
\begin{displayquote}
In this situation, I would have asked employee John Doe to work on task X, by setting a deadline over Y days, asking him to dedicate at least Z hours a day.
E.g: "John, it is important to work on the reporting feature because our client needs to delivery to their customers this new feature by the end of next month, and they are willing to let us work on the feature that we requested. If we are not able to deliver on time, they may chose to stop collaborating with us, this meaning that we might need to work on other projects that we do not like. In consequence I need you to give this task your full attention and come for help if you struggle with any impediment for more than 2 hours"
\end{displayquote}

After each exercise, the new leader was asked how is he planning to approach the next similar situations, thus obtaining a commitment that he will try a different approach, based on the recommendations,  than his previous one.

On the first iteration, the coached employee applied all of the recommendations except one: he still manifested a way to friendly attitude and language. The teams performance improved a little, but they were not yet respecting the deadlines and still focusing on the long term relationship between them. 

After the team has missed it's first set of deadlines, the leader was asked to list all the situations in the following format.
\begin{itemize}
\item What was the context
\item How were the needs communicated 
\item What was the desired outcome
\item What was the actual outcome
\item How and when could the situation have been improved.
\end{itemize}

After all the situations were evaluated, the leader observed that 
the familiar language did not inspire authority, that he did not track the progress of the team periodically in order to intervene and that he did not communicate what happens if the tasks are successful or unsuccessful.

On the next project with the same team, he adopted a firm attitude, not tolerating unannounced delays, giving public praise for every small accomplishment. The project in cause was finished before the deadline with the expected quality, thus being able to offer the team free company time to work on projects that they found interested, or learning new skills.

This approach proved that by analyzing the behavior of the leader in real context and providing advice based on experience while still leaving the coached person to make mistakes the performance of the team can be enhanced only by guiding the leader to change the way he interacts with his colleagues.

\section{Technical Dominance}
\label{sec:tech-dom}
Similar to technical leadership where the decisions were made by the leader, in the context of a team formed of junior and mid-level persons, that are lead by a senior member, with vast technical expertise an experiment was done, without enforcing technical decision but with stating all the drawbacks of the solutions from the team.

The leader has presented the desired output of the project and has let the team come up with propositions for implementation. In the same time, the leader devised an implementation plan and system architecture.  The team members were asked to present each of their propositions with the advantages and disadvantages. If the presented solution would match the solution planed by the leader, they would be interrupted by the leader with phrases like: \textit{I would do it the same way, lets go ahead with this proposition}. If the solution did not match the one planed by the leader, he than would list all the possible disadvantages (even if they would not be applicable in the current context) followed by his solution.

It has been observed that after 1-2 months the team was not proposing solutions on their own and asked the leader how he wants the features implemented. This approach worked well in terms of delivering short term projects on time, but the team did not develop any new skills. In the same time, the leader became a single point of failure for the team, as they were not able to make decisions without asking the leader. In the situations when the leader was not available, the project would come to a stop, thus creating the possibility to miss the deadlines.

While this approach helps the teams deliver short term projects, this approach is dangerous for teams that stay together for more than 6 months as mid-level member stating that they feel that their are not developing new skills and fearing that they will end up doing the same type of tasks. 

Experiments conducted with other interest that the declared ones proved to not have the desired outcome, the teams not knowing the real goal pulling in a different direction. On the other hand, the situation when the behavior of the leader was altered based on the coaches personal experience proved to be a successful approach.