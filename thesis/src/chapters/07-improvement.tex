\chapter{Improvement Experiments}
\label{chapter:improvement}
Neither leadership role is perfect for each situations. Both team leaders and people managers are necessary roles: team leaders are always needed, while the role of people managers is seen in appreciated in big companies.

At a basic level, a leader is a person who leads others. Leaders are persons who drive others, have a vision and they determined to achieve their vision. In the same time leaders do not have to be officially appointed in order to motivate and inspire others.

Based on previous experience in leadership positions 4 experiments have been carried out in order to observe their efficiency and possible uses. The first three experiments (Unofficial coaching, Behavioral coaching, Act as if  being controlled by your reports) were carried out without having an official leadership position, while the last one (Technical dominance) was carried out from an official technical leadership role.

\section{Unofficial Coaching}
\label{sec:un-coach}

The experiment was done in a context where the person being coached was a junior-middle level software engineer (based on knowledge), one year older than the coach, being in separate project team. The coach had only a software engineer role in the company, but had a 2 year previous experience in leadership. Except the coach and upper management no one was aware of the experiment being conducted.

The person being coached was manifesting an immature behavior and was always unreliable, coming into office after noon (while the expected arrival time was 9:30-10:30 AM). Coming into office so late and being involved in many projects, meant that the team was not able to gather information about the progress he made, resulting this in unnecessary delays. 

The experiment in cause was to try and solve the situations by mentoring the nonperforming employee by a colleague of the same age, but working in a different project team.

The steps of the experiments were:

\begin{enumerate}
\item Establish a personal relationship by taking smoking breaks at the same time and using the time to discuss the progress of the college classes (common point of interest outside of work) and also project development.
\item Finding out the employees in cause of the next deadlines and if they will be meat. His feelings was that there is no problem in delivering on time
\item\label{uc:st} Presenting technical coaching by recommending documentation, tutorials and giving ideas of what frameworks 
\item Comparing the employees view of the project time-line with the real one and asking his view on the discrepancies.
\item Giving firm feedback on his performance, using the \textbf{F}acts-\textbf{A}ctions-\textbf{C}orrections model \cite{abur-pm}
\item Guiding the employee to come up with a plan of actions in order to recuperate from the unpleasant situation. (Arrive in the office between 9:30-10:30 AM, Respect deadlines or announce in advance that there are impediments that will cause delays
\item Ask questions similar to \textbf{why} in order to confirm that he has understood why the action plan is important and what are the expected results.
\item\label{uc:end} Guide the employee into using a commitment language \cite{notes-to-a-software-team-leader} in order to inspire and establish engagement with the action plan.
\item Monitor the progress and appliance of the action plan.
\end{enumerate}

Unfortunately, the employee considered that his personal life was more important and that it was acceptable to go to sleep at 5:00 AM during working days and that it still is no problem that he is not available for his team to provide information about his tasks. After iterating steps \ref{uc:st}-\ref{uc:end} over a period of 2 months with this kind of discussions occurring once every two weeks.

After two months with no improvement the decision to terminate the employee was made. When asked why he did not take the recommendations into account a nonacceptance of age ( \labelindexref{Section}{sec:unnaceptance}) barrier was discovered, considering that the unofficial coach did not have enough experience in the industry, thus preferring to follow his own intuition. In the same time, being unaware of the repercussions in case the situation would not improve the employee considered they are none.

Trying to motivate employees by coaching them without a clearly presented motive and presenting only the possible gains but not the repercussions does not have the necessary leverage to improve the situation. Andrei Piti\cb{s} declared that it is important to \textit{"Do what you say and say what you do!"}. Without being open of the situations and trying to improve it from the shadow, employees do not see the real impact of his behavior and possible unpleasant results, the experiment failed thus not being a real possible improvement in leadership without improving the desire plan of action.

\section{Behavioral Coaching}
\label{sec:beh-coach}
\todo{delayed results}
\todo{do you try to change the behavior of the reports}

\section{Act as if Being Controlled by the Reports}
\label{sec:under-control}
\todo{mixed results}
\todo{do you leave the reports to think that they are the ones controlling the situation?}

\section{Technical Dominance}
\label{sec:tech-dom}
\todo{success}
\todo{doers technical enforcemnt work?}
