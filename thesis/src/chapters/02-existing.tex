\chapter{Existing Management Styles}
\label{chapter:existing}
IT companies  organize their teams and hierarchy in different ways, usually having a matrix organizational chart, dividing the leadership responsibilities between team leads, project managers and people managers. Depending on the development methodology, there could be multiple leadership positions (as described in \cite{abur-tl} and \cite{abur-pm}.

This study focuses on three main roles
\begin{itemize}
\item Technical leader (coding guru/authority)
\item Team leader/coach (mix between people manager and technical leader)
\item People manager (person outside the project team)

\end{itemize}
\section{Management Flavors}
\label{sec:flavours}
Each industry has it's own coordination and management strategy and the IT industry has a very diversified way of approaching the subject. For example: the leadership positions described below are from a pure technical point of view. All the tree situations describe ways about managing the technical employees (engineers, programmers and testers) and do not include the approaches used for the logistics, human resources or management departments.

\subsection{Technical Leader}
\label{sub-sec:tech-lead}
A technical leader is a person who usually has a lot of experience in the field (both in number of years and number of projects delivered). In \cite{abur-tl} the technical leader is also described as the \textbf{coding guru}. Although this position has proven to be very effective in projects using a waterfall approach, during the market research it has proven that leaders usually prefer to act as a technical authority for easier enforcement of their ideas.

From a top level this position can be compared to a military C.O.(commanding officer). The comparison is done by the fact that the both the C.O. and the T.L. have are responsible for coming up with solutions, communicating them to the team/squad, enforcing them if it is the case. In the same time they are the persons to blame when something goes wrong. Having this it mind, it can easily be observed that the team is viewed as an execution force with no decision responsibilities.  Although the comparison between the military field and IT industry does not have a direct one-to-one relationship the developers are usually following and implementing the T.L. ideas. On the other had the senior members of the team are sometimes consulted for designing on solution.

Technical leaders usually consider that for this position the following skills have to be fulfilled:
\begin{description}
\item [Accountability] for making the wrong decisions
\item [Good technical vocabulary] for explaining the desired solutions
\item [Fast technical decision making] for unblocking the team in dead end situations
\item [Business driven], in order to translate the clients desires into technical specifications
\item [Authority] for steering the team on track.
\item [Change request impact analysis] for adding new functionality  to a project that is already implemented or also changing the underlying integration
\item [Loyalty] to the company and immediate superiors.
\end{description}

Although a firm hand is very useful in rough situations or tough schedules, the disadvantages of this positions are represented by the fact that the leader is the only person that can guide the success or fail of the project and in the same time the team members are not encouraged to develop their soft skills. In this situation, the team members act only as programmers (awkwardly called in the market place as \textit{code monkeys} or \textit{mercenaries}).

From a task delegation point of view a T.L. will delegate only technical tasks based on the seniority level of team members. Based on the conducted interviews it has become clear that the decision to delegate a task follows in general the following steps.
\begin{enumerate}
\item Analyze the task at hand
\item Extract the necessary experience for fulfilling the task
\item Filter the potential team members to solve the task
\item Offer the task to who wants it
\item If there are more members willing to work on the task chose one one of them based on their expertise
\item If there is no one willing, chose one of the team members that has some experience in the field
\item If there is no one assigned by now, chose the one with the most spare time.
\end{enumerate}

Seeing that the T.L. is acting like an authority (or enforcer) the team members are willing to follow him if their expectations are met. Usually they expect the leader to be fair and be able to balance the workload evenly between the team members in order to ensure that all the of them have a similar load and there are no persons that have to work extra compared to the rest of the team. In the same time, the team expects that each of them get the chance to work on interesting tasks and that the lead is available and capable of providing answers to technical and business questions.

\subsection{Team Coach}
\label{sub-sec:team-coach}
\todo{describe the team coach aka team leader (R3)}
\subsection{People Manager}
\label{sub-sec:pm}
\todo{describe the people manager}
\section{Common Views}
\label{sec:common}
\todo{describe the common views}
\section{Divergent Views}
\label{sec:divergent}
\todo{describe the divergent views}
