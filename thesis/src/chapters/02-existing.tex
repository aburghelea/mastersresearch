\chapter{Existing Management Styles}
\label{chapter:existing}
IT companies  organize their teams and hierarchy in different ways, usually having a matrix organizational chart, dividing the leadership responsibilities between team leads, project managers and people managers. Depending on the development methodology, there could be multiple leadership positions (as described in \cite{abur-tl} and \cite{abur-pm}).

This study focuses on three main roles
\begin{itemize}
\item Technical leader (coding guru/authority)
\item Team leader/coach (mix between people manager and technical leader)
\item People manager (person outside the project team)

\end{itemize}
%\section{Management Flavors}
%\label{sec:flavours}
Each industry has it's own coordination and management strategy and the IT industry has a very diversified way of approaching the subject. For example: the leadership positions described below are from a pure technical point of view. All the tree situations describe ways about managing the technical employees (engineers, programmers and testers) and do not include the approaches used for the logistics, human resources or management departments.

The state of the market was analyzed by interviewing 14 leaders from three companies based in Bucharest and Brussels. And their distribution to the leadership roles is visible in \labelindexref{Figure}{img:roledist}. The lifespan of the teams coordinated distributed by the filled roles can be visible in \labelindexref{Figure}{img:lifespan}.

\fig[scale=0.7]{src/img/roledist.png}{img:roledist}{Assigned roles of interview candidates}
\fig[scale=0.7]{src/img/lifespan.png}{img:lifespan}{Team lifespan distribution according to the leadership role}
 
Although they are technical leaders, all of the interviewed candidates that have lead the same team for more than six months declared that the are acting as team coaches, thus  hinting that on the long run enforcement will start to fade and have less effects, making it necessary for the leads to focus also on the team members and not only on the projects.

\section{Technical Leader}
\label{sec:tech-lead}
A technical leader is a person who usually has a lot of experience in the field (both in number of years in the IT industry and number of projects delivered). In \cite{abur-tl} the technical leader is also described as the \textbf{coding guru}. Although this position has proven to be very effective in projects using a waterfall approach, during the market research it has proven that leaders usually prefer to act as a technical authority for easier enforcement of their ideas.

From a top level this position can be compared to a military commanding officer. The comparison is done by the fact that both the commanding officer and the team leader are responsible for coming up with solutions, communicating them to the team/squad and enforcing them if it is the case. In the same time they are the persons to blame when something goes wrong. Having this in mind, it can easily be observed that the team is viewed as an execution force with no decision responsibilities.  Although the comparison between the military field and IT industry does not have a direct one-to-one relationship the developers are usually following and implementing the technical leaders' ideas. On the other had the senior members of the team are sometimes consulted for designing the solution.

Technical leaders usually consider that for this position the following skills have to be fulfilled:
\begin{description}
\item [Accountability] for making the wrong decisions
\item [Good technical vocabulary] for explaining the desired solutions
\item [Fast technical decision making] for unblocking the team in dead-end situations
\item [Business driven], in order to translate the clients desires into technical specifications
\item [Authority] for steering the team on track.
\item [Change request impact analysis] for adding new functionality  to a project that is already implemented or also changing the underlying integration
\item [Loyalty] to the company and immediate superiors.
\end{description}

Although a firm hand is very useful in rough situations or tough schedules, the disadvantages of this positions are represented by the fact that the leader is the only person that can guide the success or fail of the project and in the same time the team members are not encouraged to develop their soft skills. In this situation, the team members act only as engineers (awkwardly called in the market place as \textit{code monkeys} or \textit{mercenaries}).

From a task delegation point of view a technical leader. will delegate only technical tasks based on the seniority level of team members. Based on the conducted interviews it has become clear that the decision to delegate a task follows in general the following steps.
\begin{enumerate}
\item Analyze the task at hand
\item Extract the necessary experience for fulfilling the task
\item Filter the potential team members to solve the task
\item Offer the task to who wants it
\item If there are more members willing to work on the task chose one of them based on their expertise
\item If there is no one willing, chose one of the team members that has some experience in the field
\item If there is no one assigned by now, chose the one with the most spare time.
\end{enumerate}

Seeing that the technical leaders is acting like an authority (or enforcer) the team members are willing to follow him if their expectations are met. Usually they expect the leader to be fair and be able to balance the workload evenly between the team members in order to ensure that all the of them have a similar load and there are no persons that have to work more than the rest of the team. In the same time, the team expects that each of them get the chance to work on interesting tasks and that the lead is available and capable of providing answers to technical and business questions.

\section{Team Coach}
\label{sec:team-coach}
From a team leadership approach the second kind of leader is the person who has medium to advanced tech skills but compensates with the fact that his soft skills (negotiation, persuasion, putting the right questions) are developed to an above average level and  is able to both give and ask for answers by correctly putting the necessary questions.

R. Osherove \cite{notes-to-a-software-team-leader} declares that``There are no experts. There is only us.'' At first sight the message that he transmits is that a team has to achieve its goal regardless of current expertise. The interesting fact was that the team leaders understood this quote as "Cut some corners, Google it and make it work somehow" while the team coaches understood it as "Let's take some time to research and understand a good way to solve this problem". During the interviews all the technical leaders declared that they would search for an article, blog or an already implemented solution to the problem to solve it as soon as possible while the team coaches and people managers preferred to invest time to research, design and prototype a couple of possible solutions and chose from the best suiting one.

The objective of the team coach is to deliver quality projects on the long term by developing the skills of the team that he/she coordinates. The team coach is not necessary the strongest person from a technical point of view but is the kind of person that can ask his team the right questions in order to guide them into coming up with a possible solution.  In order for this approach to work the team coach has to be obtain (from his superiors) and provide the team with slack time. Slack time should be interpreted as time not billable on the project, but used in the scope of learning new technologies or skills that will further help the development of the project. In agile methodologies slack time can be covert with spike tasks \cite{spike}.

In order to fill the team coach position, a person should have the following skills:
\begin{description}
\item [Good communication] for clearly stating and requesting the necessitates, status and expectations
\item [Human resource management] skills for correctly delegating the tasks and guiding the reports
\item [Motivation skills] in order to keep the team focused and determined to grow and deliver on time, scope and with good quality
\item [Persuasion skills] to convince both his managers and his reports of the decisions that he makes, without enforcing them
\item [Skill evaluation] to determine the needs of the projects, skills of the members and to develop a learning plan for the lacking knowledge
\item [Listening skills] in order to determine the needs and problems of the team and suggest resolutions
\end{description}

The major drawback of this approach is that more time is needed to reach a consensus about how things should be done and negotiations, brainstorming and meetings may be often needed. The fact that this meetings happen is not a bad thing, but it should be expected that more time will pass until the solution is implement and delivered. Compared to the technical leaders' approach, there is always a good side-effect of each completed task: part of the members (or even all of them) became aware of a new possible solution to solve the task at hand. 

After a couple of months (usually 6-12 months) the efforts of a team coach are materialized in a self-organized team that is able to handle the projects without his guidance and is able to solve their impediments either by asking the correct partners for help or by finding workarounds. In this situation, there is no need for micromanagement.

\section{People Manager}
\label{sec:pm}
While both leadership positions described above focus most on a project team, another approach would be people management (treated in detail in \cite{abur-pm}). A people manager does not have direct impact on a specific project, but focuses on the personal development of his direct reports. Although the role of a people manager can be played from different situations (human resources department, team leader, this report considers the situation of a senior employee that is outside of the project team that their direct reports act in. 

People managers are in charge of evaluating the performance of their direct reports, proposing salary adjustments and also mediating conflicts between his/hers reports and other colleagues. By being outside of the team, it is easier for a people manager to be unbiased when he makes decisions.

Keeping in mind that people managers usually focus on the personal development of his reports and also that assignations to projects not being the best match for the employee, situations where the decisions of people managers and team leaders can be in contradiction. 

Each role is based on a set of necessary skills. A list of the most important skills can be found in \labelindexref{Table}{tab:leader-skills}.

\begin{table}[htb]
	\centering
	\begin{tabular}{| p{4.2cm} | p{4cm} | p{4.2cm} |}
		\toprule
		 Technical Leader & Team Coach & People Manager \\ \midrule
	    
	    Accountable & Good developer & Visionary\\
	    Decision maker & Good motivator& Active listener \\ 
	     Trustworthy  & Trustworthy & Good motivator \\
	    Result driven & Active listener & Good assessor \\
	    Change management skills&Able to delegate & Good planner \\
	    Experienced in the domain & Conflict resolver & Inspirational \\
	    Good at explaining technical decisions &  Available and approachable & Able to read between the lines \\
	    Good knowledge of frameworks, programming languages and procedures  & & \\
\bottomrule

		\end{tabular}
		\caption{Necessary skills for the leadership roles}
		\label{tab:leader-skills}
\end{table}


From personal experience I observed that it is quite difficult to fill the role of team lead and people manager for the same direct report. The situation that I was challenged with was the fact that a mid-level member was the only person inside a team with knowledge about a specific framework while on the other hand she had no desire to continue working on the same tasks. The frustration level of my direct report when I took over the team was high enough in order for her to make the decision to leave the company. As her team coach I needed her to continue working for the next 4 months on the same type of task and in the same time I also wanted to help her learn new technologies and work on more diverse tasks. Seeing that the setup of the team did not permit to transfer her to another module, as no one could be able to do the same kind of tasks an agreement that would be in the same time a win-win and lose-lose situation. The solution was to convince her to continue working in the same way for the next two months while in the same time she would mentor another member in learning how to solve the kind of tasks that she was handling. The drawback of this approach came from the fact that the project was on a tight deadline and all the members had already been assigned to various tasks that solicited their full capacity. This meant that I was the only person that could take over from her,thus making myself over-loaded with both managerial and technical tasks and with no possibility of delegating.