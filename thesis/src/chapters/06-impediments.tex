\chapter{Impediments in People Coordination Based on the Lead - Direct Report Relationship}
\label{chapter:impediments}

Although a happy path is always desired, there are often situations were despite all the good intentions and efforts of both leaders and employees, the entire process (including the coordination processor) grinds or even comes to a stop. Impediments can come in different shapes and sizes the most common ones are: setting incorrect development objectives, misinterpretation of feedback, interpersonal conflicts \cite{abur-pm}, resistance to change \cite{abur-tl}. 

In the same time there are a couple of situations that arise from false expectations or beliefs. Some of this situations are described in the following sections.

\section{Nonacceptance of Age or Background}
\label{sec:unnaceptance}
For both type of leadership roles, acceptance from the team is important in order to set the stage for success. Being in charge of setting the direction for both project delivery and future personal development a leader must be trusted by his team, team that is able to understand and accept that although on a first look a certain decision may seem incorrect on a bigger picture that decision would fit better. 

In the cases when a team grows under the leadership of one mentor and when from a team an informal leader arises. The person in cause steps up and handles coordination in some situations, represents the team in front of clients and upper management although it is not his formal obligation. This kind of leader is recognizable by the fact that is often asked for advice by his colleagues. This fact alone proves the fact that he is already trusted by his team. At this point the informal leader can be easily appointed as a team coach or even people manager because he already acts like a mentor.

There are also situations when a leader is appointed directly and without previously working in the project team. This situations can be divided in two main categories: the professional background of the leader, measured in number of years exceeds a maturity stage (in general acceptance between 5 and 7 years), and when the leader has less than 5 years of experience or when the team colleagues have more experience in the work place than the leader.

In the case of the experienced leader, if feedback from his previous projects is good then trust is usually easily given by the reason that if \textit{he did it once good, he most probably will do it again}. If the feedback is not that good then it is possible to reach acceptance problems cause by the ego barrier \cite{sec:ego}

Among the interview candidates, 5 of them were promoted into leadership positions after less than 3.5 years of professional experience. Although from a technical point of view they were considered seniors,all  leaders experienced at least one situation where their decisions or proposals were challenged solely with the argument:
\begin{displayquote}
You haven't worked enough to backup this statements, I've done this in the past and it was ok, I don't want us to it differently.
\end{displayquote}
It the software industry (as in many other industries) there is no perfect way to do things or solve problems, there are multiple possible solutions and the ones that meet all the acceptance criteria are good enough, even if they are not perfect.

Overcoming situations like the ones above is difficult, the only option the leaders have is to either convince the team with technical arguments that his proposals are good or enforce the decisions and do a post evaluation. In the situations where enforcement is used, it is mandatory for the leader to be prepared to admit his mistakes if things do not follow the expected path. While admitting his mistakes he must give credit to the persons that advised him wrong, but he must also not undermine his own authority. For this he must be clear in the transmitted message while still using a firm tone and attitude. 

In the situations when leaders know how his/hers colleagues think, he could also guide them in coming up with a solutions and let them take both responsibility and glory for the results. But this situations usually arise after the leader is accepted in this position.
\section{The Ego Barrier}
\label{sec:ego}
In building effective teams, ego is the one of the greatest impediments from the lifespan of a team. When individuals get caught up in their own pride or even self doubt their ego creates a distorted image of their own place inside the team. 

Ego can manifest in both directions, coming from both leaders and employees and it is important to spot this situations as soon as possible and try to solve. Seeing that ego can come from both over and under evaluating the own person common signs to identify a possible ego barrier are:
\begin{description}
\item [Over promoting] the own person and trying to enforce ideas without taking into account other people's ideas. This can be interpreted as \textbf{I know better/I am always right}
\item [Avoidance] of certain members from reasons of fear because they are perceived as smarter or able to tear them down.
\item [Avoiding risks.] If a persons focuses only on low risk tasks, this means that they are not willing to fail, fearing repercussions and willing to be seen by the managers as perfectionist
\item [Refusing to admit that they were wrong] usually portraits situations when people are afraid of failing and would rather blame someone else or the circumstances for their own failure.
\end{description}

The ego barrier unfortunately leads to insisting on using the same ideas that have failed in the past, expecting different results on retrying. This in the end leads only to failure by self-sabotage.

To overcome the ego barrier it is first important to identify the cause of such a huge ego, praise must be given and employees must not fear failure.

In an collaborative environment, ego is considered a liability. Although there is nothing wrong with friendly competition, overdoing it just to feed ones own pride will render the potential of a team almost useless. In order  to achieve the companies goals, cooperation is needed and also using the talents of each individual. 

\section{Leaving Problems Uncorrected when they Appear}
\label{sec:uncorrected}
Problems can always appear and it is reckless to think that they will not appear. Project management always prepares a budget for emergency situations called  emergency or contingency fund.
There are often situations where the actions of an employee impact the work schedule, motivation or even disrupt the actions of the company. Unfortunately this kind of situations can happen and it normal to make mistakes or even fail from time to time. 

Employees should not fear failure, and they should be taught how to fail fast in order to not waste time and resources on losing directions. In the same time, it is mandatory for them to be notified when they have made a mistake. It is important to keep in mind that a manager should not punish an employee for his mistakes the first time that they happen, but rather to give feedback in order to improve the employees capabilities. Feedback should always be clear, real and be combined with action items to prevent the same situation from happening again \cite[last]. A good leader will first try first to help his reports when they fail rather than punish. 

On the other hand, if an employee continues to repeat his mistakes although he improvement actions have been tried, more drastic measures have to applied. 

If a leader oversees mistakes (even if it is the first time it happens), the team will pick up on the attitude and will start being careless and start failing, considering that it is not a problem to cause problems. The first time a problem appears, the employee in cause has to be notified so that he is aware that his mistake was seen and that it is not OK to fail, but he should never be punished. If an employee is punished from the start, then he will start to hide his future mistakes or even he will not be willing to take risks, thus inducing a fear state in the team.

Problems are natural and they should be addressed using a fair but firm language and always up front and as soon as they appear. In this way the entire team learns that leader are aware of the project life-cycle and they also see that leaders are always ready to handle the situation with the best interest of the project in mind, rather that seeking retribution from the employees that make mistakes.

Impediments are part of every leadership process, there is never the case for a happy path. Regardless of the type of problem: ego, age, or not correcting the problems as soon as they arise, good leaders should always have a plan to understand the cause first and then try to solve it. As S. Sineck describes in \cite{why}, the best way to find out the reasons of bad intentions is to ask himself and the employees in cause a simple question: \textbf{Why?}.