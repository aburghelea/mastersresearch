\chapter{Introduction}
\label{chapter:intro}
In any industry, there is no one man team. Although there are situations where for amusement purposes a single person does in parallel all the activities of an entire/team band, the scope is always the same: entertainment. But to obtain performance, a team is necessary. The situation can be easily compared to the programming concepts of threads: you have more workers that do separate tasks, but they need to be somehow synchronized or coordinated. Teams are in theory the easiest way to increase the throughput, but where there are multiple persons there also are different opinions. Although the boss was the person in charge of controlling and setting the direction, the position now shifts to leadership where control is relinquished and direction is hinted.

Leadership is present in most of the possible environments and industries, but the one that I find most interesting is the IT industry. The reason behind the interest is the fact that the IT industry is considered a spoiled industry. A spoiled industry is described as being an industry where the average salary packages are above (at least 3 to 4 times) the average salary in a country and where the number of open positions is higher than the number of possible candidates. Two companies that I have interacted with (names undisclosed) reported that the number of accepted employment offers is between 50-60\% of the total of extended offers. 

In the situation when it is hard to find new employees it becomes mandatory to have a high retention rate. Due to this fact, IT companies have started to use people management techniques in order to maintain the existing employees happy and performing.

\section{Team Leadership Concept}
\label{sec:tl}
Team Leadership in the IT industry is a concept that is centered around building products using the capabilities of the teams. Although the name says \textbf{team leadership} the emphasis is still on building \textbf{products}. Although there are multiple ways to deliver high quality products, good leadership is the an important topic that can determine the health of the organization. Team leadership is all about utilizing the strengths of each individual team member in such a way that it adds value to the team and helps bring the best outcome of the projects. The team leader role is not a \textbf{rock start position} but it is a role of great responsibility. As John C. Maxwell said, bad or good leadership can make the difference between successful or unsuccessful projects:
\begin{displayquote}
Teamwork makes the dream work, but a vision becomes a nightmare when the leader has a big dream and a bad team.
\end{displayquote}
Team leadership is not about having a crowned person in frond of the team. It's about having a group of people focused on the same goal and motivated to do their best to achieve it. It also means to set the stage for the teams success and standing in the back when they succeed. 

As described in \cite{abur-tl}, leadership is both the art and the science to be productive through the strength of the team. It's the art to motivate and focus people around a common goal in order to deliver great products.

From a technical point of view the role of team leader is responsible of delivering a high quality product on time by leveraging the knowledge of each member. As detailed in \labelindexref{Section}{sec:flavorurs} there are two main directions for leading a project oriented team.

The first one is the position of \textbf{technical leader} that will set (or even enforce) the technical solution, while the team is responsible for implementing the vision of the technical lead. The draw back of this approach comes when the visions of the members and lead are not aligned, leading to conflicts inside the team or even the need for the lead to take over tasks from the reports in order to deliver on time.

The second approach is represented by a position of \textbf{team coach} that guides the team to come up with solutions (even if they are not as he envision or even the best possible) in order to achieve the desired product. In this situation he does not enforce his solutions and helps the members develop their skills by both coming up with the design solution and implementation. His role in this case is to help and coach the team and not to enforce the technical solution. The drawback of this approach is that on the short term the time needed to develop new features may increase (depending on the expertise of the team), but while the team continues to grow it's skills this time will start to shorten.

Neither of the two approaches are favorable all the times. In case of short term projects (one to six months) a technical leadership could be favored because the time will not permit the members to learn new skills and strong guidance will be needed. On the other hand for projects that last more that half a year, a coaching approach would be better because in this case all the members learn new approaches to solve tasks and they even develop their soft-skills.

\section{People Management Concept}
\label{sec:proj-scope}
Compared to team leadership, people management does not focus directly on product delivery. The focus for people management is to bring out the best out of the biggest resource that a company has: the people. In the IT industry it is difficult to hire new employees, but it is very easy to lose them. The reason behind this is the fact that there are more open positions than new software engineers. This means that as an IT professional it is quite easy to find a well paying job in a short period of time, in the meantime it also means that as an IT recruiter it is quite difficult to find new suitable candidates for the position that are open. 

Working in a \textit{spoiled} industry, like IT, it is quite important to keep the retention rate up.  In order to achieve this, some companies have started to implement a \textbf{people management} system. A people manager is in charge of both leading by example and managing expectations. The main responsibilities of people managers are to: hire, fire, advise and motivate his direct reports. The goal of people managers is to create strong employees with a good knowledge base, mostly without direct impact on the project. The results of this actions are expected to be long term and not visible on the short run. 

Seeing that, in the vision of people managers, the end goal is not the current project, but the career path of the employees, this position is a pragmatic and results oriented type of job. The ideal solution would be to find the correct balance between the company performance and employee satisfaction, but as the interviews have shown it is not always the case, leading to tension between team leaders and people managers.
How 
As described in \cite{abur-pm}, the people manager position usually has no direct link with the human resources department, although it helps to have HR knowledge (e.g.: legal notice period, vacation days)

The long term results of each employee is not only his responsibility, it is also the managers responsibility and Peter Drucker perfectly sums it up:
\begin{displayquote}
The productivity of work is not the responsibility of the worker but of the manager.
\end{displayquote}
This means that the manager has to motivate his reports, but if everything fails he also has to make tough decisions, like firing, to ensure the success of the company.

Although both team leaders and people managers have the same end goal: the success of the company, their approaches and immediate goals seems to differ, and often end up in disagreements between the two positions.
 
\section{Motivation}
\label{sec:motivation}
During the last three years, I had the opportunity to both to fill all the leadership positions described in \labelindexref{Section}{sec:flavours}. As a technical leader I was in charge of a team of 8 juniors, as a technical coach I was lead two teams: 2 member team (both senior) and a 7 member team (mixed seniority levels) and as a people manager I had 5 direct reports with seniority levels equal or lower than mine. 

During this period I started to observe different priorities and ways of tackling day to day activities from the perspective of each position. For example, coordinating a team of junior members is quite easy, because they tend to follow your guidance, with little to no challenge. On the other hand the view of the project is not the result of an entire team, it is just the view of the lead. The team itself can be seen as an mercenary band that do the work that they are given to.
Guiding a team with mixed levels of seniority becomes a bit trickier. The senior member may hold grudges and want to shine, because they consider that the years they have spent recommends them as leaders. My biggest impediment in coordinating that team mentioned above come from the fact that I had only 3 years of professional experience, while two of the members had 10 and 15 years of experience. The situation will be described in more detail in \labelindexref{Section}{sec:unnaceptance}.

Being a people manager is quite easy as long as the objectives of the company align with the development desires of the reports. Being in charge with motivating and setting development plans can be both tricky or easy depending on the time spend with your report to get to know him. The challenges that arose here was determined that one of my direct reports was also part of the project team that I was leading. This meant that I had to make decisions in favor of both the project and the direct report, witch proved more difficult that it should have been.

The reason for elaborating this thesis was the fact that I observed both similarities and differences between the leadership positions. This differences became more obvious after seeking guidance from colleagues that were in the same positions as I was. It became clear that the interpretation of each role was different even between employees from the same company.
\section{Objectives}
\label{sec:Objectives}
The objective of this thesis is to gather information(based on the questionnaires elaborated in \cite{abur-tl} and \cite{abur-pm}  from the IT industry about the leadership roles. Based on the results from the marked, the scope is to describe what are the common/divergent views, the way that decisions are made and propose a set of improvements to the leadership approaches in order to solve the following problems:

\begin{itemize}
\item The ego barrier
\item Nonacceptance of age or background
\item Leaving problems uncorrected when they appear
\end{itemize}