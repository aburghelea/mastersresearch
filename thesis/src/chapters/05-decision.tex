\chapter{Decision Making}
\label{chapter:decision}
\todo{small chapter intro}
\section{Hiring Decisions}
\label{sec:hiring}
\todo{how do you hire somebody}
\section{Motivation Decisions}
\label{sec:motivation}
\todo{how do you motivate the employees}
\section{Delegation Decisions}
\label{sec:delegation}
\todo{how and what do you delegate}
\section{Salary Adjustment Decisions}
\label{sec:salary}
\todo{how do you decide to adjust the salary}

\section{Toxic Employee Management}
\label{sec:toxic}

Robert Yonatan, CEO at GetVoIP conducted a study inside the company in order to improve the performance of his teams. During his research \cite{toxic} he discovered 5 types of toxic employees that he names: hot mess, slacker, martyr, socialite and sociopath. Seeing that in the IT industry it is considered both difficult and costly ( 3500\$new employee \cite{cost-replace}) the interviewees were asked how the manage the slacker employees.

Yonatan describes the slacker employee as follows: They are persons that continually find ways to dodge work, relying on their team to cover for them, thus overloading them and also draining their energy. In the same time the slackers do not care what their colleagues and managers think of them, and if they are permitted to get away with it they will do it. In the same time their worst traits are considered to be: low motivation, lack of regard for deadlines, bad timekeeping, absenteeism and wasting time online. In order to get over this kind of behavior, Yonatan recommends to find the root cause for their lack of motivation (regardless if it is personal or professional), stating your expectations clearly and doing unscheduled reviews or checkups, thus forcing the employee to become accountable for his work (going against the team spirit).

When asked about slackers, the interviewees were given only the description of the slacker employee, but not the traits or possible action items. In the same time, they were not given any other clarification or guidance in answers. Keeping in mind that this was a completely open question, it was very interesting that the most of the answers were all put into three main categories: motivate, punish or take no action. The result can be seen in \labelindexref{Figure}{img:slacker}.
\fig[scale=0.6]{src/img/slacker.png}{img:slacker}{Corrective actions in the case of slacker employees}

Although they are two situations where the employees would be punished, the punishment is declared to be negative feedback or transfer to other teams.
The interesting fact is that almost half of the answers were to take no action in hope that the situation would solve by itself, the employee starting do honor his working agreement as time passes.
Although in popular belief \textit{``ignorance is bliss''}, in case of both leaders and managers prefer to motivate their reports. In the IT industry also comes from the fact that employees tend to flee the company. Different from Yonatan recommendations, in the the IT industry from Romania, no candidate considered that they should help in solving the causes of low motivation that come from personal environments. 

Concentrating only on the problems coming from the professional environment, may not always lead to improvements. Problems like conflicts between colleagues or displeasure to work with specific technologies should always be addressed and solutions should never favor one part in the detriment of the other. In the same time personal problems should be solved by the employees in cause, but a helping hand from the managers could benefit both the company and employee. In this situation the help from the work colleagues should never be intrusive or direct help, but it could be in the form of extra days off, shorter work schedules or even work from home.

In the same time the other extreme has to be handled. Workaholics (or in Yonatans' preference: the martyrs) should not be tolerated. 

In Yonatans' acceptance the martyrs are persons who usually pull long hours and preferred to have everything under their control while in the same time making it clear to all the involved parties what sacrifices they are making for the company. In the same time they tend to understatement the skills of their teammates, have a disposition to burnout, and often have a nonconstructive attitude while not knowing their own limits. 

During the interview stage, when asked about how leaders manage this kind of behavior, the interviewees were free to answer without receiving other clarifications except the description of the martyr. In \labelindexref{Figure}{img:martyr} it can be seen that the answers are can be grouped in two categories: send home or take no action. 
It is important to keep in mind that for all the situations where no action was taken, it was necessary for the delivering of the project within the agreed deadlines (a.k.a poorly planned projects).
\fig[scale=0.6]{src/img/marthyr.png}{img:martyr}{Corrective actions in the case of martyr employees}

With this in mind it can be seen that the only corrective action is to try to send the employees home. On the other hand the leaders can only recommend but not force the employees to leave, thus being a method with no real effect. 

Although this is a clear situation where both team leaders and people managers have the same approach, the current method of handling martyrs is not an effective one. This issue should be addressed because although martyrs can look like the perfect employees if seen as individuals, their actions can apply pressure on the other team members that can easily become demotived or even start to hold grudges. In case the team members have resentments, the productivity of the entire team diminishes. 

As a recommendation for this kind of behavior would be to start leading by example, trying to show a good balance between the personal and professional life. In the same time if the employee wants to work over time to gather extra benefits (only within the accepted company policy: extra free days, legal over-time remuneration) it should be presented to the team clearly that the reasons are personal and not the fact that the employee in cause is trying to pull the weight of the team or wants to become a star within the company. This approach was personally applied and having positive results on two separate occasions.

It is important to note that in the case of martyrs it was never the case to provide extra rewards (except the standard company remuneration for overtime).

Although the behaviors mentioned above are at opposite poles of the working hours, they both should be considered as negative behavior and addressed as soon as possible because both can lead to dysfunctional teams with low productivity.

\section{Firing Decisions}
\label{sec:firing}
\todo{how do you fire somebody}

The decision to stop a collaboration with an employee is always difficult to make and in some cases difficult to put in practice (e.g.: The labor law in the United States of America). Recruiting errors can always happen and in this situations drastic measures may arise. In the same time, the performance of the employees may suddenly drop for extend periods of times and in this case firing can also be handled. 

In the case of firing decisions a clear separation between the studied leadership positions was observed on the reasons for firing as follows.

\begin{description}
\item [People Managers] make the decision to fire based on the fact that an employee demotivates his colleagues.
\item [Technical Leaders] make the decision to fire based on the fact that an employee has low technical performance and is becoming a nuisance for the team as they have to pick up his/hers workload.
\item [Team coaches] make the decision to fire an employee if two criteria are met: the employee has low technical performance and if the team would be better motivated.
\end{description}

In this case it is important to observe that technical leaders focus most on the technical performance of the team and do not look at the morale of the other members while team managers act as a bridge between the tech leads and people managers. 

As a side note, 4 out 14 respondents recognized that they do not have the heart to make such a decision, preferring to escalate the situations to their direct managers and letting them handle the situation. This kind of behavior suggest a degree of professional immaturity as leadership comes with great responsibilities that need to be exerted both in good and bad situation. The candidate coordinating the biggest team and for the longest time from the studied groups even declaring:

\begin{displayquote}
My biggest failure was not firing people on time, as they managed to destroy the morale of half of my other employees and their work managed to produce significant financial losses.
\end{displayquote}

In order to avoid this situations 4 common steps have been identified that are applied by leaders, regardless of their role:

\begin{enumerate}
\item Provide a clear feedback that their performance and impact is below the expectations of the positions
\item Try to improve the performance by a joint effort between the leader and employee
\item Try to understand the root cause for the lack of performance
\item Frequently check the performance and give 
\end{enumerate}

It was also observe that people managers try to understand the cause of low performance first and they try to improve it it, while team leaders prefer to first try to solve the problems and only if that fails, they try to solve the cause. Although team leaders focus mostly on the projects, it should be important to first treat the cause not the effect as long term solutions to performance problem can not be obtained without eliminating the cause. As Simon Sinek clearly describes \cite{why} any problem should be tackled only after it is clearly understood \textbf{why} it is happening.