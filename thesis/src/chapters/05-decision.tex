\chapter{Decision Making}
\label{chapter:decision}
Making decisions in the interest of the company by leveraging the knowledge of the team is sometimes tricky. Without luck, decisions can be made only with proper analysis while keeping in mind the end scope.
\section{Hiring Decisions}
\label{sec:hiring}
Leadership involves coordination and in order to coordinate a team, it is important to first staff a team. Even though from the roles treated in this study only people managers have a clear staffing responsibility, team leaders also have a big impact in the decision to hire new employees. 

Not being a signed to a technical project, people managers are involved in hiring new employees that will be distributed inside the entire company. Provided that a possible has passed a curriculum vitae screening, they focus on three main traits:

\begin{description}
\item [Technical background.] If the candidate has a technical profile that suits the company needs (to be interpreted as open positions) and the feedback from the technical interview is positive, then he is considered a viable employee
\item [Attitude.] If the candidate is driven to learn new things and has a positive attitude he is considered a good candidate.
\item [Expectations] In the case the salary expectations of the candidate are not in the range the company can afford or are higher than seniority level, the candidate is considered a \textit{no go}.
\end{description}

On the other hand, technical leaders are involved only in technical interviews for possible candidates for the team that they are coordinating. They usually provide technical feedback to people/hiring managers and are the second step in making the decision to hire possible candidate. In most cases they have veto rights. The questionnaire candidate that have technical leadership or team coaching positions have declared that in more than 50\% of the situations, they are more strict, and rely more on the existing background of the candidate than on the potential that the candidates manifest to learn new technologies.

Team coaches tend to make the hiring decision more based on the potential to learn new technologies rather that the ones already know by the candidates. Compared to technical leaders, team coaches have a preference to hire candidates that think outside of the box and prove an analytic approach to solving unknown problems.

Hiring decision proved that they differ between people managers and team leaders. People managers focus more on attitude and financial aspect while they delegate the technical evaluation to team leaders. 

In my opinion, hiring decisions are based on a good analysis of the needs of the company and great evaluation of the candidates. In order to improve the hiring decision making, leaders have to improve their skills in requirement analysis and candidate screening.

\section{Delegation Decisions}
\label{sec:delegation}
Task delegation is an important part of the leadership activities. From a higher level it can be seen that all project related tasks are the responsibility of the team leader, being his job to distribute the workload between the team members. On the other hand, a people manager has on his plate only managerial activities that he can not usually delegate.

The observed results on delegating decision were in a clear separation between technical leaders and people managers, while team coaches have a mixed approach between the previous two positions.

For example, all technical leaders declared that they can delegate only implementation tasks after he has analyzed them following a simple pattern:
\begin{enumerate}
\item Split the tasks based on the seniority level required to handle them (senior, middle, junior)
\item Prioritize the tasks
\item Offer the tasks for each bucket based on urgency and offer them to the team members that have the least workload
\item If there is no member that wishes that manifest interest on working on the specified tasks, he picks the best suited candidate based on previous experience and technical knowledge.
\end{enumerate}

In the same time, 4 out of 7 technical leaders recognized that they have a tendency to personally work on the very urgent tasks even if this would imply over-time on their side, thus acting like martyrs (\cite{toxic}). On the other side technical leaders never delegate design decisions and planing meetings, by this proving a tendency to control every aspect of the project. 

People managers on the other side consider that their tasks can not be delegated to their direct reports. The reason for this stands in the fact that people managers are in charge of evaluating, motivating, hiring employees, thus working with sensitive and sometimes confidential information. In the same time people management is implement in companies that use employee objectives in order to provide a clear direction and performance evaluation. In order to establish trust and motivate employees, people managers declared that they sometimes delegate the choosing of objectives (in proportions of 20-30\% of the total objectives) to their reports. By applying this delegation they try to also motivate their employees to chose objectives that helps them develop on the career path that they desire.

Team coaches proved an interest on developing the team skills by using more than the members technical skills, but also relying on their soft skills. Compared to technical leaders, the technical leaders prefer to also challenge each individual to exceed his barriers by delegating tasks above his current level and then mentoring them in the solving phase. In this case they would often assign medium tasks to junior members to help them reach the next level on the seniority. But the most compelling argument for the efforts to develop the team soft skills is the fact that they often delegating leadership tasks. For example in teams with more than 4 members they often delegate an entire module to a senior member while assigning him 2 other members to coordinate. In this situation the senior member is challenged both with the technical part of the project, but is also part of the coordination efforts. If the coordination proves to be successful, this sets the path for a future leadership promotion.

Unfortunately, people managers can not delegate two much of their tasks, but technical leaders could delegate leadership tasks like the team coaches tend to do. Team coaches in this situation act like mentors and prefer to stand in the back and leave the team shine in all the situations. An improvement to the current approach that team coaches use is the fact that they could start delegation each kind of tasks until they have a team self organized to the level that they do not require a formal leadership position. As a side not, it is important to be careful to whom they delegate leadership tasks, because the situation when a member accepts a task but does not desire to do it. 
In general analytic persons and results oriented persons tend to accept leadership responsibilities only if they are formally assigned a leadership position while energetic and people oriented persons easily accept to coordinate small teams.

\section{Salary Adjustment Decisions}
\label{sec:salary}
Salary increase decisions are not always in the attributes of team leaders, but in most cases it is part of people managers responsibilities. Depending on companies people managers have either the possibility to either decide to adjust the salary or make a proposition.
Salary adjustments can be done either in the direction of increase or decrease. In the sense of salary decrease, all candidates declared that they would never make this kind of decision, because the employee would rather accept a less paying job at another company than be put in a potential humiliating situation. This proves again that the software industry is a spoiled industry were employees count on job changing rather than improving their own skills or attitude. Both type of leaders prefer to first try to motivate the nonperforming employees and if the first step fails to move onto firing, rather than diminishing the salary. 

Another possible solution that could be applied is that if after a non-performing period a performing period follows, period that in normal situations would qualify the employee for a raise, the employee to receive a one-time financial bonus.

Salary increase decisions are taken in two situations. The first one would be an alignment with the market salary levels. This kind of decisions are taken at company level and have no other fundamental that preventing an employee migration. The second case is representing by a good performance from the employees.

In the later situations technical leaders provide a technical feedback of the employee and evaluate the results compared to their initial expectations. The evaluation is usually done post-project and is not based on previously set objectives, but is usually done using the quality of the resulting product versus the expectations of the leaders from their employees. People managers propose salary increases based on previously agreed objectives and technical feedback. If the employee has above expectations from the project team and has reached most (more than 80\%) of their objectives in the agreed time line, then a percentage is recommended for the employee. The percentage may be mitigated based on the company budget and the employee may or may not receive the initially proposed increase.

Regardless of the final decision, all the candidates agreed that it is mandatory to announce a employee after the adjustment has been approved. In case an employee is announce is announce of the proposed salary increase, false expectations will be created and in the situation that the increase is not approved at the announced amount, the trust and relationship between leader and report will have to suffer. 

I strongly recommend that salaries should be kept confidential between employees while in the same time fair across seniority levels.

\section{Toxic Employee Management}
\label{sec:toxic}
Robert Yonatan, CEO at GetVoIP conducted a study inside the company in order to improve the performance of his teams. During his research \cite{toxic} he discovered 5 types of toxic employees that he names: hot mess, slacker, martyr, socialite and sociopath. Seeing that in the IT industry it is considered both difficult and costly ( 3500\$new employee \cite{cost-replace}) the interviewees were asked how the manage the slacker employees.

Yonatan describes the slacker employee as follows: They are persons that continually find ways to dodge work, relying on their team to cover for them, thus overloading them and also draining their energy. In the same time the slackers do not care what their colleagues and managers think of them, and if they are permitted to get away with it they will do it. In the same time their worst traits are considered to be: low motivation, lack of regard for deadlines, bad timekeeping, absenteeism and wasting time online. In order to get over this kind of behavior, Yonatan recommends to find the root cause for their lack of motivation (regardless if it is personal or professional), stating your expectations clearly and doing unscheduled reviews or checkups, thus forcing the employee to become accountable for his work (going against the team spirit).

When asked about slackers, the interviewees were given only the description of the slacker employee, but not the traits or possible action items. In the same time, they were not given any other clarification or guidance in answers. Keeping in mind that this was a completely open question, it was very interesting that the most of the answers were all put into three main categories: motivate, punish or take no action. The result can be seen in \labelindexref{Figure}{img:slacker}.
\fig[scale=0.7]{src/img/slacker.png}{img:slacker}{Corrective actions in the case of slacker employees}

Although they are two situations where the employees would be punished, the punishment is declared to be negative feedback or transfer to other teams.
The interesting fact is that almost half of the answers were to take no action in hope that the situation would solve by itself, the employee starting do honor his working agreement as time passes.
Although in popular belief \textit{``ignorance is bliss''}, in case of both leaders and managers prefer to motivate their reports. In the IT industry also comes from the fact that employees tend to flee the company. Different from Yonatan recommendations, in the the IT industry from Romania, no candidate considered that they should help in solving the causes of low motivation that come from personal environments. 

Concentrating only on the problems coming from the professional environment, may not always lead to improvements. Problems like conflicts between colleagues or displeasure to work with specific technologies should always be addressed and solutions should never favor one part in the detriment of the other. In the same time personal problems should be solved by the employees in cause, but a helping hand from the managers could benefit both the company and employee. In this situation the help from the work colleagues should never be intrusive or direct help, but it could be in the form of extra days off, shorter work schedules or even work from home.

In the same time the other extreme has to be handled. Workaholics (or in Yonatans' preference: the martyrs) should not be tolerated. 

In Yonatans' acceptance the martyrs are persons who usually pull long hours and preferred to have everything under their control while in the same time making it clear to all the involved parties what sacrifices they are making for the company. In the same time they tend to understatement the skills of their teammates, have a disposition to burnout, and often have a nonconstructive attitude while not knowing their own limits. 

During the interview stage, when asked about how leaders manage this kind of behavior, the interviewees were free to answer without receiving other clarifications except the description of the martyr. In \labelindexref{Figure}{img:martyr} it can be seen that the answers are can be grouped in two categories: send home or take no action. 
It is important to keep in mind that for all the situations where no action was taken, it was necessary for the delivering of the project within the agreed deadlines (a.k.a poorly planned projects).
\fig[scale=0.7]{src/img/marthyr.png}{img:martyr}{Corrective actions in the case of martyr employees}

With this in mind it can be seen that the only corrective action is to try to send the employees home. On the other hand the leaders can only recommend but not force the employees to leave, thus being a method with no real effect. 

Although this is a clear situation where both team leaders and people managers have the same approach, the current method of handling martyrs is not an effective one. This issue should be addressed because although martyrs can look like the perfect employees if seen as individuals, their actions can apply pressure on the other team members that can easily become demotived or even start to hold grudges. In case the team members have resentments, the productivity of the entire team diminishes. 

As a recommendation for this kind of behavior would be to start leading by example, trying to show a good balance between the personal and professional life. In the same time if the employee wants to work over time to gather extra benefits (only within the accepted company policy: extra free days, legal over-time remuneration) it should be presented to the team clearly that the reasons are personal and not the fact that the employee in cause is trying to pull the weight of the team or wants to become a star within the company. This approach was personally applied and having positive results on two separate occasions.

It is important to note that in the case of martyrs it was never the case to provide extra rewards (except the standard company remuneration for overtime).

Although the behaviors mentioned above are at opposite poles of the working hours, they both should be considered as negative behavior and addressed as soon as possible because both can lead to dysfunctional teams with low productivity.

\section{Firing Decisions}
\label{sec:firing}
The decision to stop a collaboration with an employee is always difficult to make and in some cases difficult to put in practice (e.g.: The labor law in the United States of America). Recruiting errors can always happen and in this situations drastic measures may arise. In the same time, the performance of the employees may suddenly drop for extend periods of times and in this case firing can also be handled. 

In the case of firing decisions a clear separation between the studied leadership positions was observed on the reasons for firing as follows.

\begin{description}
\item [People Managers] make the decision to fire based on the fact that an employee demotivates his colleagues.
\item [Technical Leaders] make the decision to fire based on the fact that an employee has low technical performance and is becoming a nuisance for the team as they have to pick up his/hers workload.
\item [Team coaches] make the decision to fire an employee if two criteria are met: the employee has low technical performance and if the team would be better motivated.
\end{description}

In this case it is important to observe that technical leaders focus most on the technical performance of the team and do not look at the morale of the other members while team managers act as a bridge between the tech leads and people managers. 

As a side note, 4 out 14 respondents recognized that they do not have the heart to make such a decision, preferring to escalate the situations to their direct managers and letting them handle the situation. This kind of behavior suggest a degree of professional immaturity as leadership comes with great responsibilities that need to be exerted both in good and bad situation. The candidate coordinating the biggest team and for the longest time from the studied groups even declaring:

\begin{displayquote}
My biggest failure was not firing people on time, as they managed to destroy the morale of half of my other employees and their work managed to produce significant financial losses.
\end{displayquote}

In order to avoid this situations 4 common steps have been identified that are applied by leaders, regardless of their role:

\begin{enumerate}
\item Provide a clear feedback that their performance and impact is below the expectations of the positions
\item Try to improve the performance by a joint effort between the leader and employee
\item Try to understand the root cause for the lack of performance
\item Frequently check the performance and give 
\end{enumerate}

It was also observe that people managers try to understand the cause of low performance first and they try to improve it it, while team leaders prefer to first try to solve the problems and only if that fails, they try to solve the cause. Although team leaders focus mostly on the projects, it should be important to first treat the cause not the effect as long term solutions to performance problem can not be obtained without eliminating the cause. As Simon Sinek clearly describes \cite{why} any problem should be tackled only after it is clearly understood \textbf{why} it is happening.

Decision making, regardless of the decision maker in the end has the same ultimate end goal: to ensure a smooth operation of a team. There are situations when both team leaders and people managers have the same reasoning and situations where the decision is done based on technical feedback from the leaders but by the people managers.

 Usually hiring and salary adjustments are done by people managers after they have the technical sign off. In this situations it is important to keep in mind that although team coaches are in the team leadership category, their approach is a hybrid between the people manager approach and technical leader approach. In the situation when firing is viewed from an abstract level and not a legal level and meaning that a member is removed from the team, both types of leaders base their decision on the same reasons: low performance, unwillingness to improve and team demoralization. 
 
If the entire legal aspect is taken into account to remove an employee from the company, then the team leaders responsibility ends in building a logical case for termination, while people managers, senior management and senior department are in charge of filling all the mandatory paperwork. In the same time, when it comes to toxic employee management both type of leaders use the same approaches, especially when it comes to slackers.

The most divergent opinions were observed in the case of delegation tasks. Usually team leaders tend to delegate technical tasks, while people managers do not have the option to delegate this kind of tasks, but delegate personal development choices.