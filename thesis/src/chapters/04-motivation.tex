\chapter{Motivations versus Enforcement}
\label{chapter:motivation}

In order to get things going, each leader has a choice. He/she can either enforce the rules or he can motivate and make the team believe in the same things as he does.

Enforcement is considered to be the situations when the leader makes the decision based on the information that he/she has and imposing his decision. During the enforcing process, the leader rarely shares the reasons he/she had for making this decision. There are cases when the decisions that were made are easily accepted by the team, especially when the team agrees with the leaders way of thinking.In the same time there are cases when the team (or parts of it) has a different view that the leader, thus being challenging for the team to start working, without understanding \textbf{why}.
From the interviews \labelindexref{Figure}{img:wayofacting} it has been observed that there pure enforcement is rarely used. 
From the two candidates that always use enforcement, one of them has lead project teams with durations between 3 and 6 months. In this situation the answer could have been counted as \textbf{Enforce only on tight deadlines}, but keeping in mind that the same team developed more projects with the same approach, the answer has been counted as \textbf{Always enforce}.

\fig[scale=0.7]{src/img/wayofacting.png}{img:wayofacting}{Preferred way of acting}

All the other candidates either prefer to always motivate or enforce only when the situation call upon it. For long lasting projects it is important for engineers to feel safe, that their work counts for the company while in the meantime the project brings additional value to the desired career path.

Motivation can come in different shapes and sizes, and it is always dependent on the person being motivated. It is mandatory for a leader to know each and every member of the team, how he/she thinks, how he/she reacts and what are the future desires. 

In order to get to know a persons way of thinking, people managers resort to one-on-one meetings and gatherings outside of the office. On-on-one meetings have the scope to establish a trustful environment where the employee can feel safe to tell his manager his struggles, desires and feedback without being scared of  being overheard by his colleagues. On the other hand out of office gatherings have the role to set the stage for a bond between team members and helping each other understand what are the priorities and preferences of each member. During this meetings, everybody should actively listen, but especially the managers. By actively listen ,he can build a profile of his report, and using a psychological model (e.g. \textbf{P}erson-\textbf{R}esults-\textbf{A}nalytical-\textbf{E}nergetic \cite{abur-pm}) he can decide on how to interact with the person in cause.

On the other hand, team leaders try to motivate their employees by assigning the correct tasks to the persons who are most interested by them. In order to build such a profile, the team coaches prefer to first ask and then challenge on each task solution, thus making associations between employees and preferred tasks.

Other motivators that leaders tend to use are: giving the best tools for the job (event if it increases costs), work from home, delegating responsibilities and the most important one, the power of example.

Even though the salary is not considered a motivator, but a hygiene factor. It is important to note that employees who feel that are not rewarded accordingly to their work, they will leave the company. 

Although enforcement usually hints a bad connotation, certain situations enforce it. This situations are usually short term projects, when debating is not an option. There are also the case when the team is composed mostly of junior members who do not have to much experience outside of collage, thus being unable to challenge the decisions of the leader with real arguments. 

Neither enforcement or motivation are suited for all situations. Depending on each case it may be better to rule by a firm hand or give the team the time to come up and own the solutions. In the same time, leaders that always rule by fear have a tendency to have a raised attrition rate, while leaders that always try to motivate take more time to deliver projects, but manage to keep the employees in the company for longer periods of time.